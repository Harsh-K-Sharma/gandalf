\documentclass[a4paper]{article}
\usepackage{txfonts}

\usepackage{graphicx}
\usepackage{amssymb}
\usepackage{supertabular}
\usepackage{array}
\usepackage[colorlinks,final]{hyperref}

\newcommand{\VERNO}{0.4.0 }
\newcommand{\var}[1]{\texttt{#1}}
\newcommand{\singlecommand}[1]{\\ \newline \indent \var{#1} \\ \newline \noindent}

\usepackage{listings}

\textheight 9.2in
\textwidth 6.2in
\oddsidemargin 0pt
\topmargin -40pt


%%%%%%%%%%%%%%%%%%%%%%%%%%%%%%%%%%%%%%%%%%%%%%%%%%%%%%%%%%%%%%%%%%%%%%%%%%%%%%
\begin{document}

\lstset{language=python,keywordstyle=\color[rgb]{0,0,1},
        commentstyle=\color[rgb]{0.133,0.545,0.133},
        stringstyle=\color[rgb]{0.627,0.126,0.941}}

\title{GANDALF v\VERNO userguide}
\author{David Hubber \& Giovanni Rosotti}

\maketitle
\tableofcontents

\newpage


%%%%%%%%%%%%%%%%%%%%%%%%%%%%%%%%%%%%%%%%%%%%%%%%%%%%%%%%%%%%%%%%%%%%%%%%%%%%%%
\section{Overview of code}
GANDALF is a new SPH and N-body code written in C++ and Python.  Although partly based on some of the algorithms and code structures used in SEREN (Hubber et al. 2011), it has been written from scratch and contains many new features and optimisations which significantly improve the speed, functionality and usability of the code.  It has been written for several purposes.
\begin{itemize}
\item GANDALF has been written with a fully object-oriented philosophy in C++.  This is to improve the maintainability of the code and also to allow the code to be easily extended in the future with as little disruption to the original code-base as possible.
\item GANDALF can be run in three different modes; as a standard C++ executable run from the command line, run from a python script or run inside an interactive python environment.  Both the script and interactive python modes include a graphical output that can be used to visualise previously run simulations, or to interactively visualise simulations as they are run on-the-fly.
\item GANDALF contains a variety of SPH algorithms such as conservative 'grad-h' SPH (Springel \& Hernquist 2002, Price \& Monaghan 2004) and Saitoh \& Makino (2013).
\item The N-body component of GANDALF contains a variety of common integration schemes such as 4th/6th-order Hermite with time-symmetric integration.
\item GANDALF can generate initial conditions at run-time, as opposed to SEREN where the initial conditions had to be prepared a priori to running the simulation.  The initial conditions may either be hard-coded into the C++ code, or set-up via a python script using numpy arrays.
\item GANDALF uses additional parameters to switch on various physics options, as well as requiring them to be activated in the Makefile.  In comparison, SEREN controlled physics options exclusively in the Makefile.  Therefore, GANDALF need not be re-compiled everytime a different setting is required.
\end{itemize}

GANDALF has been developed as a fully open-source project hosted on the site github at the address \var{https://github.com/gandalfcode/gandalf}.  Since GANDALF is currently in the beta-testing phase, it would be very helpful for the authors if new users could provide feedback of any problems with the code, or with any suggestions for improvements of existing features.  Ideally, users would submit bug reports to the GANDALF github page (\var{https://github.com/gandalfcode/gandalf/issues}) since other users can see that a bug has been identified before sending duplicate bug reports.

\newpage


%%%%%%%%%%%%%%%%%%%%%%%%%%%%%%%%%%%%%%%%%%%%%%%%%%%%%%%%%%%%%%%%%%%%%%%%%%%%%%
\section{Download \& Installation} \label{S:INSTALL}
The easiest way to download the code is using git. Just cd to the folder you prefer and type:\\
\newline
\noindent \var{git clone https://github.com/gandalfcode/gandalf.git} \\
A new folder called ``gandalf'' will be created in the current directory and a fresh version of the code will be downloaded inside it. The use of the software version control system git makes easy to stay up-to-date with the changes in the code while keeping your own version. If you want to know more about git, it is plenty of tutorials on-line. A good starting point is \var{http://git-scm.com/book/en/v2}.

\subsection{Requirements}

If you are not interested in the Python library, you need only a C++ compiler. In order to use all features in GANDALF, the following programs and libraries must be installed :
\begin{itemize}
\item C++ compiler
\item Python 2.7
\item swig compatible with python 2.7
\item matplotlib compatible with python 2.7
\item numpy
\item scipy
\end{itemize}

To reduce the number of needed libraries, we ship together with GANDALF the following packages:
\begin{itemize}
\item pyparsing
\item cmd2
%\item ffmpeg (for generating movies automatically)
\end{itemize}
We thank the authors for writing this software and releasing them under the MIT license.

Finally, in order to generate movies through the Python library, you need to have ffmpeg installed and invokable from the command line.

\subsection{Linux}
All of these programs/libraries can be found in most standard Linux installations, and if not, will be available to download from most package managers (e.g. apt-get, yum, pacman).

\subsection{Mac OS X}
For Mac users, all programs can be installed with either fink, MacPorts or homebrew.  It is preferable that all are downloaded with just the one package manager in order to ensure they are compatible and function together correctly. Other options for installation are the Anaconda or Enthought Python distributions. A good reference which compares the available possibilities is the \href{http://python4astronomers.github.io/installation/recommended_options.html}{Python4Astronomers website}.



%%%%%%%%%%%%%%%%%%%%%%%%%%%%%%%%%%%%%%%%%%%%%%%%%%%%%%%%%%%%%%%%%%%%%%%%%%%%%%%
\subsection{Command-line code compilation}

To compile GANDALF only as a C++ executable to be run from the command line without visualisation via the python program, first set the chosen C++ compiler in the Makefile and type \\
\newline
\noindent \var{make executable} \\

\noindent The code is compiled and linked with the chosen C++ compiler and all python components are ignored.  If you do not have python installed on your system, or are having trouble getting the python components to function correctly, then the C++ executable can still be compiled and run stand-alone.


%%%%%%%%%%%%%%%%%%%%%%%%%%%%%%%%%%%%%%%%%%%%%%%%%%%%%%%%%%%%%%%%%%%%%%%%%%%%%%%
\subsection{Python code compilation}
To compile all code elements including the python components, then complete the following steps :
\begin{itemize}

\item Install all python-related programs listed in Section \ref{S:INSTALL}

\begin{itemize}
\item For Linux machines, python is usually installed by default.  The additional python libraries should be easily installed using either the main package manager,  {\var pip} or {\var easy\_install}.
\item For Mac OS X, python is installed by default.  However, it does not have directly compatible versions of all the required libraries.  Therefore, it is required to install an additional version of python 2.7 using a 3rd party package manager like fink, Macports or homebrew.
\end{itemize}

\item Add the location of the folder containing the main GANDALF directory to the \var{PYTHONPATH} environment variable.  If you are using bash or related shells, then add the line \\
\newline
\noindent \var{export PYTHONPATH=XXX/YYY:\$PYTHONPATH} \\

\noindent to your \var{.bashrc} (or \var{.bash\_profile} on a Mac) script where \var{XXX/YYY/gandalf} is the absolute path of the GANDALF directory (note that you must not include ``gandalf'' in the folder name!). If you are using csh, tcsh or related shells, then add \\
\newline
\noindent \var{setenv PYTHONPATH ``XXX/YYY:\$PYTHONPATH''} \\

\noindent to your relevant shell configuration file. Remember that you have to close and reopen the shell for this change to take effect, or to \var{source} the shell configuration file.

\item Set the required version of python in your Makefile (See Section \ref{S:MAKEFILE}).  Since operating systems usually have more than one version of python installed, it is important to ensure that \var{make} uses the correct version when compiling the code.  This is set using the \var{PYTHON} variable in the Makefile.

Also, \var{make} requires the location of the python and numpy libraries.  In most cases, \var{make} will be able to locate these libraries automatically (see description of Makefile options).  However, if there is a problem, or you wish to use an alternative version of these libraries installed elsewhere on your system, then these can be set by the \var{PYLIB} and \var{NUMPY} variables in the Makefile.


\item Finally, to compile all components of the code, including the python libraries with swig, type \\
\newline
\noindent \var{make} or \var{make all} \\

\end{itemize}

\newpage


%%%%%%%%%%%%%%%%%%%%%%%%%%%%%%%%%%%%%%%%%%%%%%%%%%%%%%%%%%%%%%%%%%%%%%%%%%%%%%%
\section{Basic usage}
GANDALF can be run in three principle modes.
\begin{itemize}
\item Command-line mode, where the code is run as a C++ executable via the command line with a parameters file that selects which code options are used,
\item via a python script where the code is run in a python environment with various plotting options available as well as running the code,
\item in interactive mode where all the code and plotting options can be run directly by typing them into an interactive python shell.
\end{itemize}



%%%%%%%%%%%%%%%%%%%%%%%%%%%%%%%%%%%%%%%%%%%%%%%%%%%%%%%%%%%%%%%%%%%%%%%%%%%%%%%
\subsection{Command-line mode}

\noindent To run a simulation with the C++ executable on the command line, type : \\
\newline
\var{./gandalf PARAMSFILE} \\
\newline
where PARAMSFILE is the name of the parameters file for that simulation.  If that parameters file does not exist in the current directory, or contains invalid parameter options, then the program will quit citing an error message.  If the parameters file exists, then the code will parse the file, set-up the simulation and then run to completion.



%%%%%%%%%%%%%%%%%%%%%%%%%%%%%%%%%%%%%%%%%%%%%%%%%%%%%%%%%%%%%%%%%%%%%%%%%%%%%%%
\subsection{Running python scripts}

\noindent Any python script using the GANDALF python API can be run directly from the command-line as any regular python script.  To run the script PYTHONSCRIPT.py, type \\
\newline
\var{python PYTHONSCRIPT.py} \\
\newline
or, depending on the default version of python on your system (e.g. installing matplotlib with fink on Mac OS X), \\
\newline
\var{python2.7 PYTHONSCRIPT.py} \\
\newline
The code will remain inside the python environment until the script is fully executed.  Python typically checks that the entire file has valid syntax before running each command.



%%%%%%%%%%%%%%%%%%%%%%%%%%%%%%%%%%%%%%%%%%%%%%%%%%%%%%%%%%%%%%%%%%%%%%%%%%%%%%%
\subsection{Interactive mode}

\noindent To open the interactive viewer, type : \\
\newline
\var{python analysis/gandalf\_interpreter.py} \\
\newline
or, depending on the default version of python on your system (e.g. installing matplotlib with fink on Mac OS X), \\
\newline
\var{python2.7 analysis/gandalf\_interpreter.py} \\
\newline
\noindent The code should open with a splash screen containing the code title followed by a command prompt of the form '\var{gandalf >'}'.  To run a simulation defined by a parameter file, then type \\
\newline
\var{newsim PARAMSFILE} \\
\newline
\noindent and then \\
\newline
\var{run} \\
\newline
\noindent The current simulation can be plotted at any point by using simple commands such as, for example, \\
\newline
\var{plot x y} \\
\newline
\noindent For a list on available commands, type \var{help} in the command line.  For more detailed information on the functionality of a particular command, type \var{help command}.  For more information on interactive python commands, see Section \ref{S:PYTHONSCRIPT}.

\newpage



%%%%%%%%%%%%%%%%%%%%%%%%%%%%%%%%%%%%%%%%%%%%%%%%%%%%%%%%%%%%%%%%%%%%%%%%%%%%%%
\section{Makefile options} \label{S:MAKEFILE}
The GANDALF Makefile is used to select options which are used to compile the code with.  If the user wishes to change any compile-time options, the code must be recompiled from scratch by typing {\var make clean} and then \var{make}.  For users of SEREN, the GANDALF Makefile has been simplified with many options either not present (since various specialist algorithms have not been implemented) or have been transfered to the parameters file.  This has been done in order to make it less likely that the wrong Makefile options are used in simulations, and also to stop the need to recompile the code completely so often when using slightly different options.

\begin{itemize}

\item CPP : C++ compiler \\
\begin{tabular}{ll}
g++ & : GCC C++ compiler \\
icpc & : Intel C++ compiler \\
clang & : CLANG compiler (Mac OS X only)
\end{tabular}

\item PYTHON : name of python command-line executable (e.g. python, python2.7)

\item COMPILER\_MODE : Set compiler flags for production or debug runs \\
\begin{tabular}{ll}
DEBUG & : Set all debug compiler options, including flags to use gdb debugger and full warning output (-Wall) \\
STANDARD & : Standard optimisation options (-O3) \\
FAST & : -O3 + fast flag options.  Uses 'potentially unsafe' fast maths optimisation.
\end{tabular}

\item PRECISION : Floating point precision \\
\begin{tabular}{ll}
SINGLE & : 32-bit precision floating point variables \\
DOUBLE & : 64-bit precision floating point variables
\end{tabular}

\item OPENMP : Activate OpenMP directives during compilation (0 or 1)

\item OUTPUT\_LEVEL : Amount of output produced by code \\
\begin{tabular}{ll}
0 & : No additional output \\
1 & : Minimal output of main-loop routines \\
2 & : Code routine marker output for all steps
\end{tabular}

\item DEBUG\_LEVEL : Amount of extra debug checking done by code \\
\begin{tabular}{ll}
0 & : No extra debugging computations and output \\
1 & : Activate assert debug statements in code \\
2 & : Activate extra (expensive) debugging computations in code
\end{tabular}

\item FFTW : Include FFTW (Fast Fourier transform) library for initial conditions (0 or 1).  The FFTW\_LIBRARY and FFTW\_INCLUDE variables should contain the library links and include directory if different from the standard Linux directories (otherwise leave blank).

\item GSL : Include GSL (GNU Scientific library) which is required for Ewald forces (0 or 1). The GSL\_LIBRARY and GSL\_INCLUDE variables should contain the library links and include directory if different from the standard Linux directories (otherwise leave blank).

\item PYLIB : Path to directory that includes prefered python libraries.  If left blank, {\var make} will use python routines to locate the libraries automatically.

\item NUMPY : Path to directory that includes prefered numpy libraries.  Ase with PYLIB, If left blank, {\var make} will use python routines to locate the library locations automatically.

\item GTEST : Path to directory containing Google test suite library for unit testing (Currently set witu environment variable)

\end{itemize}

\newpage



%%%%%%%%%%%%%%%%%%%%%%%%%%%%%%%%%%%%%%%%%%%%%%%%%%%%%%%%%%%%%%%%%%%%%%%%%%%%%%
\section{Parameter options} \label{S:PARAMETERS}

\noindent With the exception of a few compiled-time options that are selected in the Makefile, all physics and code feature options are controlled from the parameters file, including the dimensionality, simulation type and SPH options.  We list here all possible parameters, including all possible options for those parameters with a limited choice.  We note that a parameters file need not contain a value for each parameter, in which case the default value is taken.


%%%%%%%%%%%%%%%%%%%%%%%%%%%%%%%%%%%%%%%%%%%%%%%%%%%%%%%%%%%%%%%%%%%%%%%%%%%%%%
\subsection{Main simulation parameters}

\begin{itemize}

\item \var{ndim} : Simulation dimensionality ($1$, $2$ or $3$)

\item \var{sim} : Simulation type \vspace{0.1cm} \\
\begin{tabular}{ll}
sph               & = SPH (+ N-body) algorithm (default : `grad-h' SPH) \\
gradhsph          & = `grad-h' SPH simulation (+ N-body) \\
sm2012sph         & = Saitoh \& Makino (2012) SPH (+ N-body) \\
meshlessfv        & = Meshless Finite-Volume algorithm (default : `mfvmuscl') \\
mfvmuscl          & = Meshless FV MUSCL integration simulation \\
mfvrk             & = Meshless FV Runge-Kutta integration \\
nbody             & = N-body only simulation
\end{tabular}

\item \var{nbody} : Main N-body integration algorithm \\
\begin{tabular}{ll}
lfkdk        & = 2nd-order Leapfrog kick-drift-kick \\
lfdkd        & = 2nd-order Leapfrog drift-kick-drift \\
hermite4     & = 4th-order Hermite scheme \\
hermite4ts   & = Time-symmetric 4th-order Hermite scheme
\end{tabular}

\item \var{ic} : Simulation initial conditions \vspace{0.1cm} \\
\begin{tabular}{ll}
file              & = Load initial conditions from external file \\
bb                & = Boss-Bodenheimer (1979) test \\
binary            & = Simple binary star test \\
binaryacc         & = Binary accretion test \\
blastwave         & = Blastwave test \\
bondi             & = Spherically symmetric Bondi accretion test \\
box               & = Create a uniform box of gas \\
cdiscontinuity    & = Contact discontinuity test \\
ewaldcylinder     & = Cylinder for 1D Ewald gravity test \\
ewaldsine         & = Sinusoidal density for 3D Ewald gravity test \\
ewaldslab         & = Slab for 2D Ewald gravity test \\
gresho            & = Gresho-Chan vortex test \\
khi               & = Kelvin-Helmholtz instability test \\
noh               & = Noh problem initial conditions \\
plummer           & = Plummer sphere test \\
quadruple         & = Simple hierarchical quadruple star test \\
sedov             & = Sedov blast-wave test \\
shearflow         & = Shear flow test \\
shocktube         & = Shocktube test \\
soundwave         & = 1D soundwave perturbation \\
sphere            & = Uniform density sphere \\
spitzer           & = Ionised bubble Spitzer expansion test \\
triple            & = Simple hierachical triple star test \\
turbcore          & = Turbulent spherical, self-gravitating core \\
python            & = Generate initial conditions from python
\end{tabular}

\item \var{run\_id}  : Simulation run id string

\item \var{in\_file} : Input filename (when \var{ic = file})

\item \var{in\_file\_form} : Format of initial conditions file \\
\begin{tabular}{ll}
column           & = Simple column data format \\
sf/seren\_form   & = SEREN ASCII format \\
su/seren\_unform & = SEREN binary format
\end{tabular}

\item \var{out\_file\_form} : Format of outputted snapshot files \\
\begin{tabular}{ll}
column           & = Simple column data format \\
sf/seren\_form   & = SEREN ASCII format \\
su/seren\_unform & = SEREN binary format
\end{tabular}

\item \var{tend} : Termination time of the simulation (given in {\var tunit}s)

\item \var{tmax\_wallclock} : Maximum allowed wallclock time for simulation before being terminated

\item \var{dt\_snap} : Snapshot time interval (given in {\var tunit}s)

\item \var{tsnapfirst} : Time of first snapshot (given in {\var tunit}s)

\item \var{Nstepsmax} : Maximum no. of steps in simulation before termination

\item \var{noutputstep} : Frequency of screen output (in units of integer steps)

\item \var{ndiagstep} : No. of complete block steps between diagnostic output

\item \var{nrestartstep} : No. of full block steps before producing restart dump

\item \var{litesnap} : Output `lite' snapshots (for generating movies)? ($0$ or $1$)

\item \var{dt\_litesnap} : Lite snapshot time interval (given in {\var tunit}s)

\item \var{tsnapfirst} : Time of first lite snapshot (given in {\var tunit}s)

\end{itemize}



%%%%%%%%%%%%%%%%%%%%%%%%%%%%%%%%%%%%%%%%%%%%%%%%%%%%%%%%%%%%%%%%%%%%%%%%%%%%%%
\subsection{Unit parameters} \label{SS:UNITPARAMS}

\begin{itemize}
\item \var{dimensionless} : Are all quantities dimensionless? ($0$ or $1$)

\item \var{routunit} : Position unit \vspace{0.1cm} \\
\begin{tabular}{ll}
pc/kpc/mpc & = parsec/kiloparsec/megaparsec \\
au         & = astronomical unit \\
r\_sun     & = Solar radius \\
r\_earth   & = Earth radius \\
cm/m/km    & = centimetre/metre/kilometre
\end{tabular}

\item \var{moutunit} : Mass unit \vspace{0.1cm} \\
\begin{tabular}{ll}
m\_sun          & = Solar mass \\
m\_jup/m\_earth & = Jupiter mass/Earth mass \\
g/kg            & = gram/kilogram
\end{tabular}

\item \var{toutunit} : Time unit \\
\begin{tabular}{ll}
yr/myr/gyr & = year/megayear/gigayear \\
day        & = day \\
sec        & = second
\end{tabular}

\item \var{voutunit} : Velocity unit \\
\begin{tabular}{ll}
cm\_s/m\_s/km\_s & = centimetres/metres/kilometres per second \\
au\_yr           & = astronomical units per year
\end{tabular}

\item \var{aoutunit} : Acceleration unit \\
\begin{tabular}{ll}
cm\_s2/m\_s2/km\_s2 & = cm/m/km per second squared \\
au\_yr2             & = astronomical units per year squared
\end{tabular}

\item \var{rhooutunit} : Density unit \\
\begin{tabular}{ll}
m\_sun\_pc3 & = Solar masses per parsec cubed \\
kg\_m3      & = kilogrammes per metre cubed \\
g\_cm3      & = grammes per centimetre cubed
\end{tabular}

\item \var{sigmaoutunit} : Column/surface density unit \\
\begin{tabular}{ll}
kg\_m2      & = kilogrammes per meter squared
\end{tabular}

\item \var{pressoutunit} : Pressure unit \\
\begin{tabular}{ll}
Pa    & = pascals/newtons per square metre \\
bar   & = bars
\end{tabular}

\item \var{foutunit} : Force unit \\
\begin{tabular}{ll}
N    & = newtons \\
dyn  & = dynes
\end{tabular}

\item \var{Eoutunit} : Energy unit \\
\begin{tabular}{ll}
J/GJ      & = joules/gigajoules \\
erg       & = ergs \\
eV        & = electron volts \\
10\^40erg & = $10^{40}$ ergs
\end{tabular}

\item \var{momoutunit} : Momentum unit \\
\begin{tabular}{ll}
m\_sunkm\_s  & = Solar masses kilometres per second \\
m\_sunau\_yr & = Solar masses A.U. per year \\
kgm\_s       & = Kilogram metres per second \\
gcm\_s       & = Gram centimetres per second
\end{tabular}

\item \var{angmomoutunit} : Angular momentum unit \\
\begin{tabular}{ll}
m\_sunkm2\_s  & = Solar masses kilometres squared per second \\
m\_sunau2\_yr & = Solar masses A.U. squared per year \\
kgm2\_s       & = Kilogram metres squared per second \\
gcm2\_s       & = Gram centimetres squared per second
\end{tabular}

\item \var{angveloutunit} : Angular velocity unit \\
\begin{tabular}{ll}
rad\_s & = Radians per second
\end{tabular}

\item \var{dmdtoutunit} : Mass (accretion) rate unit \\
\begin{tabular}{ll}
m\_sun\_myr  & = Solar masses per megayear \\
m\_sun\_yr   & = Solar masses per year \\
kg\_s        & = kilogrammes per second \\
g\_s         & = grammes per second
\end{tabular}

\item \var{Loutunit} : Luminosity unit \\
\begin{tabular}{ll}
L\_sun       & = Solar luminosity \\
W            & = watts \\
erg\_s       & = ergs per second
\end{tabular}

\item \var{kappaoutunit} : Mass opacity unit \\
\begin{tabular}{ll}
m2\_kg       & = metre squared per kilogram \\
cm2\_g       & = centimetre squared per gram
\end{tabular}

\item \var{Boutunit} : Magnetic field unit \\
\begin{tabular}{ll}
tesla     & = tesla \\
gauss     & = gauss
\end{tabular}

\item \var{Qoutunit} : Charge unit \\
\begin{tabular}{ll}
C    & = coulomb \\
e    & = electron charge
\end{tabular}

\item \var{Jcuroutunit} : Current density unit \\
\begin{tabular}{ll}
C\_s\_m2    & = coloumb per second per metre squared
\end{tabular}

\item \var{uoutunit} : Specific energy unit \\
\begin{tabular}{ll}
J\_kg  & = Joules per kilogram \\
erg\_g & = ergs per gram
\end{tabular}

\item \var{dudtoutunit} : Heating rate unit \\
\begin{tabular}{ll}
J\_kg\_s  & = Joules per kilogram per second \\
erg\_g\_s & = ergs per gram per second
\end{tabular}

\item \var{tempoutunit} : Temperature unit \\
\begin{tabular}{ll}
K & = Kelvin
\end{tabular}

\end{itemize}



%%%%%%%%%%%%%%%%%%%%%%%%%%%%%%%%%%%%%%%%%%%%%%%%%%%%%%%%%%%%%%%%%%%%%%%%%%%%%%
\subsection{Integration and timestep parameters}

\begin{itemize}

\item \var{accel\_mult} : Acceleration timestep multiplier

\item \var{courant\_mult} : Courant timestep multiplier

\item \var{nbody\_mult} : N-body timestep multiplier

\item \var{subsys\_mult} : Sub-system N-body timestep multiplier

\item \var{Nlevels} : No. of initial timestep levels

\item \var{level\_diff\_max} : Maximum allowed SPH neighbour timestep difference

\item \var{sph\_single\_timestep} : Constrain all SPH particles to a single timestep level

\item \var{nbody\_single\_timestep} : Constrain all N-body particles to a single timestep level


\end{itemize}




%%%%%%%%%%%%%%%%%%%%%%%%%%%%%%%%%%%%%%%%%%%%%%%%%%%%%%%%%%%%%%%%%%%%%%%%%%%%%%
\subsection{Hydrodynamical parameters}

\begin{itemize}
\item \var{hydro\_forces} : Compute hydro forces?  ($1$ or $0$)

\item \var{gas\_eos} : Gas particles equation-of-state \\
\begin{tabular}{ll}
energy\_eqn & = Solve energy equation \\
isothermal  & = Isothermal EOS \\
barotropic  & = Barotropic EOS (i.e. for mimicing isothermal + adiabatic phase during protostellar collapse) \\
barotropic2  & = Similar to barotropic, but using discrete power laws rather than smooth change \\
rad\_ws     & = EOS relating to Stamatellos et al. (2007) cooling method 
\end{tabular}


\item \var{energy\_integration} : Energy integration scheme (only applicable if solving the energy equation) \\
\begin{tabular}{ll}
null        & = Energy equation not integrated separately \\
rad\_ws     & = Integrate energy terms using Stamatellos et al. (2007) method
\end{tabular}

\item \var{energy\_mult} : Explicit energy integration timestep multiplier

\item \var{gamma\_eos} : Ratio of specific heats for gas

\item \var{temp0} : (Isothermal) temperature (isothermal or barotropic EOS)

\item \var{mu\_bar}    : Mean gas particle mass (in units of hydrogen mass)

\item \var{rho\_bary} : Adiabatic density turnover in barotropic EOS (in g/cm\^3)

\item \var{eta\_eos}   : Polytropic exponent (for barotropic EOS)

\item \var{radws\_table} : Name of EOS file for Stamatellos et al. (2007) cooling method

\item \var{temp\_ambient} : Ambient temperature (for rad\_ws method)


%\item \var{Acool} : Simple cooling rate factor

%\item \var{u\_eq} : Equilibrium internal energy (for simple cooling law)

\end{itemize}



\subsection{SPH parameters}

\begin{itemize}

\item \var{sph\_integration} : SPH particle integration scheme \\
\begin{tabular}{ll}
lfkdk   & = 2nd-order Leapfrog kick-drift-kick \\
lfdkd   & = 2nd-order Leapfrog drift-kick-drift
\end{tabular}

\item \var{kernel} : SPH kernel function \\
\begin{tabular}{ll}
m4        & = M4 Cubic spline kernel \\
quintic   & = Quintic spline kernel \\
gaussian  & = Gaussian kernel (truncated at 3h)
\end{tabular}

\item \var{tabulated\_kernel} : Tabulate kernel function  ($1$ or $0$)

\item \var{h\_fac} : Particles-per-smoothing length factor (eta in papers)

\item \var{h\_converge} : Smoothing length iteration convergence tolerance

\end{itemize}





%%%%%%%%%%%%%%%%%%%%%%%%%%%%%%%%%%%%%%%%%%%%%%%%%%%%%%%%%%%%%%%%%%%%%%%%%%%%%%
\subsection{Artificial viscosity parameters}

\begin{itemize}

\item \var{avisc} : Artificial viscosity options \\
\begin{tabular}{ll}
none  & = No artificial viscosity \\
mon97 & = Monaghan (1997) viscosity
\end{tabular}

\item \var{acond} : Artificial conductivity options \\
\begin{tabular}{ll}
none        & = No artificial conductivity \\
price2008   & = Price (2008) conductivity \\
wadsley2008 & = Wadsley et al. (2008) conductivity
\end{tabular}

\item \var{time\_dependent\_avisc} : Morris \& Monaghan time-dependent viscosity ($1$ or $0$)

\item \var{alpha\_visc} : (Maximum) value of alpha viscosity parameter

\item \var{alpha\_visc\_min} : Minimum value of alpha for time-dependent viscosity

\item \var{beta\_visc} : Value of beta viscosity as a multiple of alpha


\end{itemize}



%%%%%%%%%%%%%%%%%%%%%%%%%%%%%%%%%%%%%%%%%%%%%%%%%%%%%%%%%%%%%%%%%%%%%%%%%%%%%%
\subsection{Meshless finite-volume parameters}

\begin{itemize}

\item \var{riemann\_solver} : Riemann solver in FV scheme \\
\begin{tabular}{ll}
exact  & = Exact Riemann solver (e.g. Toro 1999) \\
hllc & = HLLC approximate Riemann solver
\end{tabular}

\item \var{slope\_limiter} : Slope limiter for TVD condition \\
\begin{tabular}{ll}
null          & = No limiting \\
zeroslope     & = Set all slopes to zero (effectively 1st order Godunov) \\
balsara2004   & = Balsara (2004) slope-limiter \\
springel2009  & = Original AREPO (Springel 2009) slope limiter \\
tess2011      & = TESS slope limtier \\
gizmo         & = Original GIZMO paper (Hopkins 2015) slope limiter \\
minmod        & = simplified implementation of minmod slope limiter
\end{tabular}

\item \var{zero\_mass\_flux} : Use Meshless-Finite Mass scheme to prevent mass-flux between particles?   ($1$ or $0$)
\item \var{static\_particles} : Use static particles (Eulerian approach)?  ($1$ or $0$)

\end{itemize}



%%%%%%%%%%%%%%%%%%%%%%%%%%%%%%%%%%%%%%%%%%%%%%%%%%%%%%%%%%%%%%%%%%%%%%%%%%%%%%
\subsection{Gravitational parameters}

\begin{itemize}

\item \var{self\_gravity} : Compute gravitational forces?   ($1$ or $0$)

\item \var{kgrav} : Direction of (external) gravitational acceleration ($0$, $1$ or $2$)

\item \var{grav\_kernel} : Form of gravitational softening \\
\begin{tabular}{ll}
mean\_h   & = Mean smoothing length softening
\end{tabular}

\item \var{external\_potential} : External gravitational potential \\
\begin{tabular}{ll}
none     & = No external potential \\
vertical & = Constant gravitational field \\
plummer  & = Plummer background potential
\end{tabular}

\item \var{avert} : Veritcal (constant) gravitational acceleration

\item \var{rplummer\_extpot} : Background Plummer potential radius

\item \var{mplummer\_extpot} : Background Plummer potential mass

\end{itemize}



%%%%%%%%%%%%%%%%%%%%%%%%%%%%%%%%%%%%%%%%%%%%%%%%%%%%%%%%%%%%%%%%%%%%%%%%%%%%%%
\subsection{Neighbour searching and tree parameters}

\begin{itemize}

\item \var{neib\_search} : Neighbour searching algorithm \vspace{0.1cm} \\
\begin{tabular}{ll}
bruteforce & = Brute-force (i.e. summation over all particles) \\
kdtree     & = Balanced kd-binary tree \\
octtree    & = Barnes-Hut octal tree
\end{tabular}

\item \var{gravity\_mac} : Gravity-tree cell-opening criteria (N.B. always defulats to geometric for now) \vspace{0.1cm} \\
\begin{tabular} {ll}
geometric & = Standard Barnes-Hut geomtric opening angle criterion \\
eigenmac  & = Compute eigenvalues of quadrupole moment tensor for MAC (Hubber et al. 2011)
\end{tabular}

\item \var{multipole} : Multipole expansion for tree-gravity \vspace{0.1cm} \\
\begin{tabular} {ll}
monopole       & = Monopole-only terms for cell gravity \\
quadrupole     & = Include quadrupole moment terms for cell gravity \\
fast\_monopole & = Compute monpoles more efficiently using Taylor expansion about cell COM
\end{tabular}

\item \var{Nleafmax} : Maximum no. of particles allowed in tree leaf cell

\item \var{ntreebuildstep} : Integer steps inbetween tree re-builds

\item \var{ntreestock} : Integer steps inbetween tree re-stocks

\item \var{thetamaxsqd} : Maximum tree gravitational walk opening angle (squared)

\item \var{macerror} : MAC error tolerance for individual cells

\end{itemize}





%%%%%%%%%%%%%%%%%%%%%%%%%%%%%%%%%%%%%%%%%%%%%%%%%%%%%%%%%%%%%%%%%%%%%%%%%%%%%%
\subsection{N-body parameters}

\begin{itemize}

\item \var{sub\_systems} : Identify and integrate sub-systems separately? ($0$ or $1$)

\item \var{sub\_system\_integration} : Main N-body integration algorithm \\
\begin{tabular}{ll}
lfkdk        & = 2nd-order Leapfrog kick-drift-kick \\
hermite4     & = 4th-order Hermite scheme \\
hermite4ts   & = Time-symmetric 4th-order Hermite scheme
\end{tabular}

\item \var{Npec} : No. of ${\rm P(EC)^n}$ iterations in time-symmetric scheme (if non time-symmetric scheme is used, automatically sets to $1$)

\item \var{nbody\_softening} : Use SPH kernel-softening between star particles? ($0$ or $1$)

%\item \var{perturbers} : Use perturbers when calculating sub-system quantities ($1$ or $0$)

\item \var{binary\_stats} : Output binary statistics? ($1$ or $0$)

\item \var{nsystembuildstep} : Integer steps inbetween re-building the sub-system tree.

\item \var{gpefrac} : Maximum fraction of total gravitational potential energy from external sources to allow sub-system.

%\item \var{gpesoft} : Maximum fraction of total gravitational potential energy from ...

%\item \var{gpehard} : Maximum fraction of total gravitational potential energy from ...

\end{itemize}





%%%%%%%%%%%%%%%%%%%%%%%%%%%%%%%%%%%%%%%%%%%%%%%%%%%%%%%%%%%%%%%%%%%%%%%%%%%%%%
\subsection{Sink particle parameters}

\begin{itemize}

\item \var{sink\_particles}   : Do stars/sinks accrete?  ($0$ or $1$)

\item \var{create\_sinks}     : Create new sink particles?  ($0$ or $1$)

\item \var{smooth\_accretion} : Use smooth accretion?  ($0$ or $1$)

\item \var{fixed\_sink\_mass} : Fixed sink mass, even when accreting? ($0$ or $1$)

\item \var{extra\_sink\_output} : Extra output of sink particles? ($0$ or $1$)

\item \var{rho\_sink}         : Sink particle creation density (in cgs units)

\item \var{alpha\_ss}         : Sunyaev-Shakura alpha for smooth disc accretion

\item \var{sink\_radius}      : Sink particle radius (in units of smoothing length)

\item \var{smooth\_accrete\_frac} : Smooth accretion instantaneous accretion mass frac.

\item \var{smooth\_accrete\_dt} : Smooth accretion instantaneous accretion timestep frac.

\item \var{sink\_radius\_mode} : How to calculate new sink radius \\
\begin{tabular}{ll}
hmult   & = sink radius a multiple of SPH particle smoothing length \\
fixed   & = sink radius is fixed for all new sinks \\
\end{tabular}


\end{itemize}



%%%%%%%%%%%%%%%%%%%%%%%%%%%%%%%%%%%%%%%%%%%%%%%%%%%%%%%%%%%%%%%%%%%%%%%%%%%%%%
\subsection{Radiation parameters}

\begin{itemize}

\item \var{radiation} : Main radiation algorithm used \\
\begin{tabular}{ll}
none            &= No radiation field \\
ionisation      &= Multiple source ionising radiation
\end{tabular}

\item \var{Nphoton}   : No. of photon packets (for Monte-Carlo radiation transport)
\item \var{mu\_ion}   : Mean-gas particle mass for ionised gas
\item \var{temp\_ion} : Temperature of ionised gas
\item \var{arecomb}   : Recombination coefficient (in cgs units)
\item \var{Ndotmin}   : No. of ionising photons per second
\item \var{NLyC}      : No. of ionising photons per second

\end{itemize}



%%%%%%%%%%%%%%%%%%%%%%%%%%%%%%%%%%%%%%%%%%%%%%%%%%%%%%%%%%%%%%%%%%%%%%%%%%%%%%
\subsection{Boundary parameters}

\begin{itemize}

\item \var{boundary\_lhs[0]} : Boundary conditions for LHS of x-dimension
\item \var{boundary\_rhs[0]} : Boundary conditions for RHS of x-dimension
\item \var{boundary\_lhs[1]} : Boundary conditions for LHS of y-dimension
\item \var{boundary\_rhs[1]} : Boundary conditions for RHS of y-dimension
\item \var{boundary\_lhs[2]} : Boundary conditions for LHS of z-dimension
\item \var{boundary\_rhs[2]} : Boundary conditions for RHS of z-dimension
For all boundaries: \\
\begin{tabular}{ll}
open     & = open boundaries (i.e. extends to infinity) \\
periodic & = periodic wrapping between LHS \& RHS boundary \\
wall     & = wall at boundary (i.e. reflection of particles)
%mirror   & = mirror boundary (i.e. 'ghost' reflections)
\end{tabular}

\item \var{boxmin[0]} : Location of LHS x-boundary
\item \var{boxmax[0]} : Location of RHS x-boundary
\item \var{boxmin[1]} : Location of LHS y-boundary
\item \var{boxmax[1]} : Location of RHS y-boundary
\item \var{boxmin[2]} : Location of LHS z-boundary
\item \var{boxmax[2]} : Location of RHS z-boundary


\end{itemize}




%%%%%%%%%%%%%%%%%%%%%%%%%%%%%%%%%%%%%%%%%%%%%%%%%%%%%%%%%%%%%%%%%%%%%%%%%%%%%%
\subsection{Initial conditions parameters}

\begin{itemize}

\item \var{particle\_distribution} : Particle configuration when generating uniform density fluids on the fly \\
\begin{tabular}{ll}
random         &= Particle positions generated with random number generator \\
cubic\_lattice &= Particle positions generated on a uniform cubic lattice \\
hexagonal\_lattice &= Particle positions generated on a hexagonal closed-packed array
\end{tabular}

\item \var{smooth\_ic} : Smooth any particle quantities around discontinuities
\item \var{com\_frame} : Translate ICs to COM frame before starting simulation
\item \var{regularise\_particle\_ics} : Regularise particle initial conditions before main simulation
\item \var{Nreg}         : No. of regularisation iterations
%\item \var{field\_type}  : ..
%\item \var{gridsize}     : ..
\item \var{Nhydro}       : No. of hydrodynamical particles
\item \var{Nhydromax}    : Maximum no. of hysrodynamical particles
\item \var{Nstar}        : No. of star particles
\item \var{Nstarmax}     : Maximum no. of star particles

\item \var{Nlattice1[0]} : No. of ptcls on lattice 1 in x-dimension
\item \var{Nlattice1[1]} : No. of ptcls on lattice 1 in y-dimension
\item \var{Nlattice1[2]} : No. of ptcls on lattice 1 in z-dimension
\item \var{Nlattice2[0]} : No. of ptcls on lattice 2 in x-dimension
\item \var{Nlattice2[1]} : No. of ptcls on lattice 2 in y-dimension
\item \var{Nlattice2[2]} : No. of ptcls on lattice 2 in z-dimension

\item \var{vfluid1[0]}   : x-velocity of fluid 1
\item \var{vfluid1[1]}   : y-velocity of fluid 1
\item \var{vfluid1[2]}   : z-velocity of fluid 1
\item \var{vfluid2[0]}   : x-velocity of fluid 2
\item \var{vfluid2[1]}   : y-velocity of fluid 2
\item \var{vfluid2[2]}   : z-velocity of fluid 2

\item \var{rhofluid1}    : Density of fluid 1
\item \var{rhofluid2}    : Density of fluid 2

\item \var{press1}       : Pressure of fluid 1
\item \var{press2}       : Pressure of fluid 2

\item \var{amp}          : Amplitude of applied perturbation
\item \var{lambda}       : Wavelength of applied perturbation

\item \var{kefrac}       : Fraction of energy that is kinetic (Sedov test)
\item \var{radius}       : Radius of cloud
\item \var{angvel}       : Angular velocity of cloud (in radians per second)
\item \var{mcloud}       : Mass of cloud

\item \var{rplummer}     : Plummer radius
\item \var{mplummer}     : Total mass of plummer sphere

\item \var{rstar}        : (Softening) radius of star particles

\item \var{cdmfrac}      : Fraction of mass in cdm particles
\item \var{gasfrac}      : Fraction of mass in gas particles
\item \var{starfrac}     : Fraction of mass in star particles

\item \var{m1}           : Mass of star 1
\item \var{m2}           : Mass of star 2
\item \var{m3}           : Mass of star 3
\item \var{m4}           : Mass of star 4

\item \var{abin}         : Semi-major axis of binary orbit 1
\item \var{abin2}        : Semi-major axis of binary orbit 2
\item \var{ebin}         : Orbital eccentricity of binary orbit 1
\item \var{ebin2}        : Orbital eccentricity of binary orbit 2

\item \var{phirot}       : Phi Euler rotation angle
\item \var{thetarot}     : Theta Euler rotation angle
\item \var{psirot}       : Psi Euler rotation angle
\item \var{vmachbin}     : Speed of binary COM through ambient gas

\item \var{alpha\_turb}  : Turbulent energy (as multiple of gravitational energy)
\item \var{power\_turb}  : Power spectrum slope of initial turbulent velocity field

\item \var{asound}       : Sound speed
\item \var{zmax}         : ??

\end{itemize}


%%%%%%%%%%%%%%%%%%%%%%%%%%%%%%%%%%%%%%%%%%%%%%%%%%%%%%%%%%%%%%%%%%%%%%%%%%%%%%%
\subsection{Random numer generator parameters}

\begin{itemize}

\item \var{rand\_algorithm} : Random number generator algorithm \\
\begin{tabular}{ll}
none     & = No algorithm selected; use intrinsic generator \\
         & \,\,\,(not recommended since this is system dependent and is not reproducable on different machines) \\
xorshift & = Xorshift generator (see Numerical recipes, Ed 3, Chapter 7 for details)
\end{tabular}

\item \var{randseed}        : Random number seed

\end{itemize}





%%%%%%%%%%%%%%%%%%%%%%%%%%%%%%%%%%%%%%%%%%%%%%%%%%%%%%%%%%%%%%%%%%%%%%%%%%%%%%
\subsection{MPI parameters}

\begin{itemize}

\item \var{mpi\_decomposition}  : Mode of MPI decomposition \\
\begin{tabular}{ll}
kdtree & = Use simple KD-tree decomposition
\end{tabular}
\item \var{pruning\_level\_min} : Minimum level to prune exported trees
\item \var{pruning\_level\_max} : Maximum level to prune exported trees

\end{itemize}



%%%%%%%%%%%%%%%%%%%%%%%%%%%%%%%%%%%%%%%%%%%%%%%%%%%%%%%%%%%%%%%%%%%%%%%%%%%%%%%
\subsection{Python viewer parameters}

\begin{itemize}

\item \var{dt\_python} : Time interval (in seconds) between view window updates

\end{itemize}


\newpage




%%%%%%%%%%%%%%%%%%%%%%%%%%%%%%%%%%%%%%%%%%%%%%%%%%%%%%%%%%%%%%%%%%%%%%%%%%%%%%%
\section{Generating initial conditions}

There are three main ways of generating initial conditions in GANDALF.

%One important difference between SEREN and GANDALF is that initial conditions can be generated on-the-fly in GANDALF, i.e. while running the code, unlike SEREN where initial conditions always had to be generated before the simulation.  While they are still situations where it is appropriate to prepare initial conditions prior to running the simulations, it is more convenient to generate the initial conditions at the same time as running the simulation. It's also possible to generate the initial conditions in python (see next section).


\subsection{`On-the-fly' initial conditions}

It is possible to generate initial conditions `on-the-fly' using internal subroutines in GANDALF.  At present, the following initial conditions are included in the code : \newline

\noindent SPH simulations : \\
\newline
\begin{tabular}{ll}
- bb             &: Boss-Bodenheimer test \\
- binaryacc      &: Binary accretion simulation \\
- bondi          &: Spherically symmetric Bondi accretion test \\
- box            &: Create uniform box \\
- cdiscontinuity &: Contact-discontinuity test \\
- ewald          &: Several tests of Ewald gravity \\
- khi            &: Kelvin-helmholtz instability \\
- noh            &: Noh shock test \\
- plummer        &: Plummer sphere (stars + gas, or just gas) \\
%- rti            &: Rayleigh-Taylor instability \\
- sedov          &: Sedov blastwave test \\
- shearflow      &: Shearing flow test \\
- shocktube      &: Simple two-fluid shocktube test \\
- soundwave      &: Simple 1D sound-wave perturbation test \\
- sphere         &: Create uniform density sphere \\
- spitzer        &: Spitzer expansion of HII region test \\
- turbcore       &: Create uniform density core with turbulent velocity field
\end{tabular}
\newline
\newline

\noindent N-body simultions : \\
\newline
\begin{tabular}{ll}
- binary    &: Simple circular binary system test \\
- burrau    &: Burrau Pythagorean test \\
- figure8   &: Simple 3-body figure-8 test \\
- plummer   &: Plummer sphere (stars + gas, or just stars) \\
- quadruple &: Simple hierarchical quadruple system test \\
- triple    &: Simple hierarchical triple system test
\end{tabular}


\subsection{Load from external file}
GANDALF can load two main file formats; simple ASCII column format and SEREN format.

\subsection{Generate inside python script}
GANDALF can generate initial conditions if using python scripts.  The initial conditions can be first set in numpy arrays and then imported into the C++ code in order to perform the simulation.  More information is given about this method, including detailed examples, in Section \ref{SS:PYTHONTUTORIAL}.


\newpage



%%%%%%%%%%%%%%%%%%%%%%%%%%%%%%%%%%%%%%%%%%%%%%%%%%%%%%%%%%%%%%%%%%%%%%%%%%%%%%%
\section{GANDALF Python library} \label{S:PYTHONSCRIPT}

The GANDALF Python library can be used for many purposes,
\begin{itemize}
\item To load in previously run simulations for analysis and producing visualisation
\item To load in multiple previously run simulations for comparisons
\item To prepare often-run simulations including the analysis in a single script
\item To run batches of simulations (e.g. a parameter study) with a single controlling script
\item To run or analyse simulations interactively through the interpreter
\item To generate initial conditions directly in python (instead of via a file or in C++) and run the simulation
\end{itemize}
It can therefore provide an easier entry into using SPH and N-body than other codes which require more investigation of the code mechanics and file formats before even basic simulations can be run.

%While the numerical part of GANDALF  is implemented in C++ for efficiency, we provide bindings to Python to control the simulation and a large set of analysis and plotting tools to inspect the results. Therefore, from Python you are able to set up the simulation (including initial conditions), run it, load existing snapshots and plot quantities.

There are two main ways to use the Python side of GANDALF; (a) a python script, and  (b) the interpreter, that works similarly to a shell (e.g. bash, csh).  Both methods understand a specific set of commands to load a simulation and plot quantities, although in principle using a Python script is more powerful %We provide a large set of commands to do standard operations when running and analyzing a simulation, so that this solution can be satisfactory where you need to quickly analyze a simulation.
% The second way is writing your own python script. You still have access to the same commands available in the interpeter, but combining the power of the python language you are free to do much more.
since you can access the raw particle data (via swig) and compute additional particle properties.  The Python files for both the library and the interpreter and contained in the \var{gandalf/analysis} sub-directory.



%%%%%%%%%%%%%%%%%%%%%%%%%%%%%%%%%%%%%%%%%%%%%%%%%%%%%%%%%%%%%%%%%%%%%%%%%%%%%%%
\subsection{Python script usage}
The GANDALF Python library contains a number of modules that can be imported in to provide the desired functionality.
\begin{itemize}
\item \lstinline[columns=fixed]{gandalf.analysis.facade} \\
\noindent This module is the main front-end to the GANDALF Python library and contains the Python-wrapped C++ executable and all functions required to set-up and run simulations.  This module must always be imported in GANDALF Python scripts.
\item \lstinline{gandalf.analysis.compute} \\
\noindent This module contains all extra and user-defined routines for computing important quantities from the simulation snapshot data.  Currently contains routines for computing centre-of-mass properties, L1 error norm (when provided with an analytical solution) and the Lagrangian radius.
\item \lstinline{gandalf.analysis.data\_fetcher} \\
This module contains routines for exporting data from the C++ code to Python, generating custom data (e.g. time evolution) from snapshot data and generating user-defined data quantities from raw particle data.
\end{itemize}



%%%%%%%%%%%%%%%%%%%%%%%%%%%%%%%%%%%%%%%%%%%%%%%%%%%%%%%%%%%%%%%%%%%%%%%%%%%%%%%
\subsection{Interpreter usage}

%The interpreter is located in the analysis/ folder in the main directory of GANDALF.
To start the interpreter, type \\
\newline
\noindent \var{python analysis/gandalf\_interpreter.py}. \\
\newline
A list of commands available for the python interpeter can be printed by typing the 'help' command.  Furthermore, typing 'help command' gives more information on the chosen command.  The commands in the interpreter are principally the same as those used in the script.  The main difference is the scripts call functions with arguments in parenthesis serparated by commas, whereas the interpreter works like a command-line tool with arguments separated by spaces.  For example, the plot command in the python script will appear like \lstinline{plot(`x',`y')} whereas the interpreter equivalent would be typed as \var{plot x y}.


%Like in a shell, you can use the interpreter interactively or you can write little scripts for it (that is, texts files with a list of the commands that you want to be executed). \textbf{TODO: must actually implement reading from a script}. Any script can be read in and processed by typing : \var{read SCRIPTNAME}.

%The interpreter provides commands for creating new simulations, loading a simulation and do the plotting. The next section contains some examples to show how to use it.


%%%%%%%%%%%%%%%%%%%%%%%%%%%%%%%%%%%%%%%%%%%%%%%%%%%%%%%%%%%%%%%%%%%%%%%%%%%%%%%
\subsection{Commands}
To be done



%%%%%%%%%%%%%%%%%%%%%%%%%%%%%%%%%%%%%%%%%%%%%%%%%%%%%%%%%%%%%%%%%%%%%%%%%%%%%%%
%\subsection{Python interpreter examples}



%%%%%%%%%%%%%%%%%%%%%%%%%%%%%%%%%%%%%%%%%%%%%%%%%%%%%%%%%%%%%%%%%%%%%%%%%%%%%%%
%\subsection{Write your Python script}

%All the commands available in the interpreter can also be used in your Python script. Just note the slight difference in syntax (since they are function, you need to put the arguments inside parentheses and they must be comma-separated). To use them, you need to import the module facade.py in the analysis/ folder. We recommend to add it to your PYTHONPATH to make the import simpler. Each function in facade comes with documentation, so you can inspect that to know what a function does.

%Here are the same examples from the previous section, rewritten to show how to use them in a Python script:
%\textbf{add examples}

%In addition, in your script you can also perform other tasks that are not possible from the interpreter, since you have a full programming language at your disposal. For example:
%\begin{itemize}

%\item You can modify from Python the parameters of the simulation before running it. This is very useful to do parameter studies, without generating in advance many different parameter files. Just remember to change the run\_id for each one of the simulations that you want to run, otherwise all the snapshots will be overwritten.

%\textbf{Example}

%\item It is possible to generate the initial conditions from Python. You need to create numpy arrays containing the quantities for the SPH particles (position, velocity, mass, internal energy) and you can import them in the code. An example follows:
%\textbf{Write the example!}

%\item You can define your own quantities to add to the plots in addition to the built-in ones. \textbf{Not implemented yet}

%\item It is possible to get access to the raw data of the simulation, in order to compute whatever quantity you need. \textbf{Actually, this is not possible yet in an easy way... I need to write a proper wrapper around the C++ method which is too low level. Remember -> give 'snap' attribute to the array to save reference counting. Implement it!}

%\item Everything else that you can think. The point is, it's GANDALF that takes care of reading the snapshots, getting the data, and all this boring parts; you can concentrate on what to do with the data and on the science, and forget about the details.

%\end{itemize}

%%%%%%%%%%%%%%%%%%%%%%%%%%%%%%%%%%%%%%%%%%%%%%%%%%%%%%%%%%%%%%%%%%%%%%%%%%%%%%%

%\section{Structure of the python code}

%If you just want to use the python code, the previous section is enough. But if you want to develop some new feature, or you are just curious to know how we implemented it, you can go on reading.

%The functions defined in facade act mostly as wrappers around lower levels of abstraction. Sometimes they also make two different portions of the code communicate. The three main lower-level portions of the code are:
%\begin{itemize}

%\item SimBuffer. This class is responsible to keep track of all the simulations you have loaded, and the snapshots that are available for each one of them, either on disk or in memory (if you are running it). To eliminate the overhead of reading from the disk as much as possible (which is particularly important with network mounted filesystems), data is cached in memory once read. When all the cache is used, the buffer will look for objects that are no longer used since a long time and deallocate them to make space. The algorithm used is LRU (least recently used), which unfortunately is not scan resistant. However, we think that scans along all the snapshots are not something frequently enough to be a problem. The size in memory of the cache is a parameter that you can set; by default it is 1 GB. \textbf{Give to the user the possibility to change it}. Always use the functions in the buffer when you want to have the raw simulation/snapshot object, rather than reading them manually; in addition to saving you a lot of coding, this ensures that you are using caching and speeding up the reading.

%\item Plotting process. In order to have the figures responsive while the simulation is running, the plotting part is done in another process. A queue is used to make the main process (the one that responds to the user commands, and where the simulation runs) communicate with the plotting one. Two kinds of objects are transferred by this queue: the data to plot, and the commands to execute (see next element). The plotting process executes a loop: it reads what is there in the queue, and executes the commands if there is any. Then sleeps for a while, letting the user interact with the plots.

%There is a potential performance improvement in the transferring of the data along the queue. Internally, python uses the pickle module to serialize the array, which is quite slow. Using zeromq, it would be possible to copy the binary data, which would be just a memcopy and therefore very fast compared to the plotting itself. Another possible improvement is to store the data used, and avoid transferring it again each time a plot is updated.

%\item Commands. The functions in facade that interact with the plotting process create a command object and send them through the queue. A command is a class that contains a method that gets called by the plotting process when the object is received. Therefore, it's the command itself that knows what to do when it's called, rather than the plotting process. This separates the roles of the plotting process from the ones of the commands. Some of the commands also contain methods for getting the data from the simulation (this happens in the main process, where the data is stored).


%\end{itemize}

\newpage
%%%%%%%%%%%%%%%%%%%%%%%%%%%%%%%%%%%%%%%%%%%%%%%%%%%%%%%%%%%%%%%%%%%%%%%%%%%%%%%
\subsection{GANDALF Python library tutorial} \label{SS:PYTHONTUTORIAL}
The GANDALF Python library contains a variety of commands that are used to create and run new simulations, load in old simulations from the disk, analyse and plot results and more.  A full list of all of these commands is given in Appendix A (to be written).  The same information can be found in the source code file \var{gandalf/analysis/facade.py}.  Also, when running the interpreter, information on interactive commands can be obtained by typing \var{help} for a full list of available commands, or \var{help command} for information on that particular command.

We provide here a short tutorial demonstrating from the most basic to the more advanced functionality of the Python library.  These examples are also contained in the \var{gandalf/examples} sub-directory and should run as typed, although it is also useful for the user to write the examples themselves if unfamiliar with python syntax. \\
\newline


%%%%%%%%%%%%%%%%%%%%%%%%%%%%%%%%%%%%%%%%%%%%%%%%%%%%%%%%%%%%%%%%%%%%%%%%%%%%%%%
\subsubsection{Example 1 - Creating and running a new simulation from a parameter file}
This first example demonstrates how to set-up and run a new simulation from a given parameters file (nominally called \var{params.dat} in these examples).  \\

%\begin{tabular}{p{14.0cm}}
%\var{from gandalf.analysis.facade import *} \\
%\var{sim = newsim(`adsod.dat')}            \\
%\var{setupsim()} \\
%\var{run()} \\
%\end{tabular} \\
%\newline

\lstinputlisting{../examples/example01.py}

\noindent The first command \singlecommand{from gandalf.analysis.facade import *} loads in all definitions and functions from the GANDALF python frontend, \var{facade.py}.  Note that this assumes that the \var{PYTHONPATH} environment variable has been correctly set-up.  Otherwise, the full absolute path of the \var{facade.py} file must be given.  The second command \singlecommand{sim = newsim(`adsod.dat')} reads in the parameters file \var{'adsod.dat'} and creates a new simulation from those parameters and returns a Python object, \var{sim}, which can be used to refer to that simulation in Python (N.B. the user can use any name for this object, as long as it does not clash with another GANDALF or Python object).  In this case, it is assumed the parameters file is selecting an initial conditions generator in the main code to generate the particles in the simulation.  In order to fully set-up the simulation so it is ready to be run, the command \singlecommand{setupsim()} must be run.  Finally, to run the simulation to completion, we must execute the command \singlecommand{run()}  The code will run until the specified endtime and then exit the python environment.
\newpage


%%%%%%%%%%%%%%%%%%%%%%%%%%%%%%%%%%%%%%%%%%%%%%%%%%%%%%%%%%%%%%%%%%%%%%%%%%%%%%%
\subsubsection{Example 2 - Creating a simulation and modifying a parameter before running}
This example demonstrates how to modify parameters in Python that have been loaded in from a given parameters file. \\

%\begin{tabular}{p{14.0cm}}
%\var{from gandalf.analysis.facade import *} \\
%\var{sim = newsim(`adsod.dat')} \\
%\var{sim.SetParam(`tend',1.5)} \\
%\var{setupsim()} \\
%\var{run()} \\
%\end{tabular} \\

\lstinputlisting{../examples/example02.py}



\noindent The first two lines of this example are the as the first example, where we import the GANDALF Python frontend and then create a new simulation from the parameters file \var{`adsod.dat'}.  Once loaded in, we can now modify parameters with the command \singlecommand{sim.SetParam(parameter\_name,new\_value)}  In this case, we set the parameter \var{tend} (the total simulation run-time) to the value \var{2.0} with \lstinline{sim.SetParam(`tend',2.0)}.  Once all parameters have been modified, then we can set-up the simulation fully with \lstinline{setupsim()} and run it with \lstinline{run()}.  Note that if we attempt to modify any parameters AFTER calling \lstinline{setupsim()}, then the code will terminate with an exception.


%%%%%%%%%%%%%%%%%%%%%%%%%%%%%%%%%%%%%%%%%%%%%%%%%%%%%%%%%%%%%%%%%%%%%%%%%%%%%%%
\subsubsection{Example 3 - Generating initial conditions using internal routines without a parameters file}
In this example, we will create and run a simulation without needing to load-in an external parameters file. \\

%\begin{tabular}{p{14.0cm}}
%\var{from gandalf.analysis.facade import *} \\
%\var{sim = newsim(ndim=2,sim=`sph')} \\
%\var{sim.SetParam(`sph',`gradh')} \\
%\var{sim.SetParam(`ic',`sedov')} \\
%\var{sim.SetParam(`run\_id',`SEDOV1')} \\
%\var{sim.SetParam(`Nlattice[0]',64)} \\
%\var{sim.SetParam(`Nlattice[1]',64)} \\
%\var{sim.SetParam(`boxmin[0]',-1.0)} \\
%\var{sim.SetParam(`boxmin[1]',-1.0)} \\
%\var{sim.SetParam(`boxmax[0]',1.0)} \\
%\var{sim.SetParam(`boxmax[1]',1.0)} \\
%\var{sim.SetParam(`dimensionless',1)} \\
%\var{sim.SetParam(`Nlevels',10)} \\
%\var{sim.SetParam(`tend',0.5)} \\
%\var{sim.SetParam(`tsnapfirst',0.0)} \\
%\var{sim.SetParam(`dt\_snap',0.1)} \\
%\var{setupsim()} \\
%\var{run()} \\
%\end{tabular} \\

\lstinputlisting{../examples/example03.py}



\noindent In this example, we call the \var{newsim} function passing the dimensionality of the simulation instead of a parameter file.  Therefore the line \lstinline{sim = newsim(ndim=2,sim=`sph')} sets up a 2-dimensional SPH simulation, but is otherwise unspecified because no other parameters have been set.  Next, we can set all the parameters that would have otherwise been set in the parameters file.  In this example, we set up an SPH simulation with initial conditions to perform the Sedov blast-wave test (\lstinline{sim.SetParam(`ic',`sedov')}); (See Section \ref{S:PARAMETERS} for a full description of all parameters).  Once all parameters have been set, we can fully set-up and run the simulation with \lstinline{setupsim()} and \lstinline{run()}.
\newpage


%%%%%%%%%%%%%%%%%%%%%%%%%%%%%%%%%%%%%%%%%%%%%%%%%%%%%%%%%%%%%%%%%%%%%%%%%%%%%%%
\subsubsection{Example 4 - Running a simulation and plotting results to screen and file}
In this example, we show how to plot particle data interactively during a live simulation. \\

%\begin{tabular}{p{14.0cm}}
%\var{from gandalf.analysis.facade import *} \\
%\var{sim = newsim(`glass.dat')} \\
%\var{sim.SetParam(`dt\_python',2.0)} \\
%\var{setupsim()} \\
%\var{plot(`x',`y')} \\
%\var{run()} \\
%\var{savefig(`figure.eps')} \\
%\var{block()} \\
%\end{tabular} \\

\lstinputlisting{../examples/example04.py}



\noindent In this simulation, we will run a live simulation while also plotting particle data in a matplotlib window as the simulation is still running.  We first create the simulation as in previous examples with \var{newsim}.  We then set the value of the \var{dt\_python} which is the time between python/matplotlib window updates.  After the \var{setupsim()} command, we can then plot the initial conditions from the simulation using the command \singlecommand{plot(x,y)} where \var{x} and \var{y} are the x- and y-axis plotting variables (N.B. these can be any variables, not necessarily x and y. e.g. \var{plot(`y',`z')}).  If we now run the simulation with \var{run()}, the simulation will compute as normal while updating the plot window every \var{dt\_python} seconds (in this casem every 2 seconds).  After the simulation has finished, we can save the final plot to file with the command \singlecommand{savefig(filename)} where \var{filename} is the intended filename {\it including} the file extension.  matplotlib automatically determines the correct file format from the extension  (e.g. typing \lstinline{savefig(`fig1.png')} will automatically save the window to a png file without need for additional arguments).


%%%%%%%%%%%%%%%%%%%%%%%%%%%%%%%%%%%%%%%%%%%%%%%%%%%%%%%%%%%%%%%%%%%%%%%%%%%%%%%
\subsubsection{Example 5 - Reading a simulation from disc and plotting several snapshots}
In this example, we load in a previously run simulation from the disk and generate various plots. \\

%\begin{tabular}{p{14.0cm}}
%\var{from gandalf.analysis.facade import *} \\
%\var{loadsim('ADSOD1')} \\
%\var{plot(`x',`rho',snap=0)} \\
%\var{savefig(`snap1.eps')} \\
%\var{plot(`x',`rho',snap=1)} \\
%\var{savefig(`snap2.eps')} \\
%\var{plot(`x',`rho',snap=0)} \\
%\var{addplot(`x',`rho',snap=1)} \\
%\var{savefig(`snap12.eps')} \\
%\var{block()} \\
%\end{tabular} \\

\lstinputlisting{../examples/example05.py}



\noindent Instead of a creating a new simulation with the \var{newsim} function, we will load in a previous run simulation with the command \singlecommand{loadsim(runid)} where \var{runid} is the simulation run identification string.  Note, this assumes that all files from that simulation are present in the folder, including the \var{runid.param} file which contains all the parameters used to run the simulation.  In this case, we load a simulation with the runid \var{ADSOD1} with the command \lstinline{loadsim(`ADSOD1')}.  We can then plot particular snapshots of the simulation using the \var{plot} command with the optional \var{snap} argument.  Therefore to plot the first snapshot, type \var{plot(`x',`y',snap=0)}, to plot the second snapshot, type \var{plot(`x','y',snap=1)}, etc..  Note that Python uses C-style indexing, i.e. starting from zero, unlike other conventions like in FORTRAN where indexing starts from one.  If we wish to overplot one snapshot over another, we first plot the first snapshot and then use the command \singlecommand{addplot(`x',`y',snap=1)} to add the second snapshot plot over the first one.  This behaviour can also be replicated by adding the \var{overplot} optional boolean argument to \var{plot}, i.e. \lstinline{plot(`x',`y',snap=1,overplot=True)}.


%%%%%%%%%%%%%%%%%%%%%%%%%%%%%%%%%%%%%%%%%%%%%%%%%%%%%%%%%%%%%%%%%%%%%%%%%%%%%%%
\subsubsection{Example 6 - Reading  and plotting multiple simulations}
In this example, we show how to read in multiple simulations to make comparison plots. \\

%\begin{tabular}{p{14.0cm}}
%\var{from gandalf.analysis.facade import *} \\
%\var{sim0 = loadsim(`ADSOD1')} \\
%\var{sim1 = newsim(`adsod.dat')} \\
%\var{sim1.SetParam(`run\_id',`ADSOD2')} \\
%\var{sim1.SetParam(`avisc',`none')} \\
%\var{setupsim()} \\
%\var{run()} \\
%\var{plot(`x',`rho',sim=0,snap=2)} \\
%\var{addplot(`x',`rho',sim=1,snap=2)} \\
%\var{savefig(`sod12.eps')} \\
%\var{block()} \\
%\end{tabular} \\

\lstinputlisting{../examples/example06.py}




\noindent Multiple simulations can be loaded into GANDALF using the \var{loadsim} command.  However, they must be returned to objects with different names; in this case, we name them \var{sim0} and \var{sim1} (to follow the C-style numbering convention).  A snapshot from a particular simulation can be plotted by adding the optional argument \var{sim} to the plot command, i.e. \lstinline{plot(`x',`rho',sim=0,snap=0)} to plot the first snapshot in the first simulation.  The second simulation is then over-plotted with the command \lstinline{addplot(`x',`rho',sim=1,snap=0)}.  In this example, we load in an old simulation (as run in example 1) and then run a new simulation (the same initial conditions but without any artificial viscosity) and then plot the results on the same window and to an eps file.


%%%%%%%%%%%%%%%%%%%%%%%%%%%%%%%%%%%%%%%%%%%%%%%%%%%%%%%%%%%%%%%%%%%%%%%%%%%%%%%
\subsubsection{Example  7 - Overplotting the analytical solution with the simulation results}
In this example, we demonstrate how to overplot analytical solutions over particle plots. \\

%\begin{tabular}{p{14.0cm}}
%\var{from gandalf.analysis.facade import *} \\
%\var{sim = newsim(`adsod.dat')} \\
%\var{sim.SetParam(`dt\_python',2.0)} \\
%\var{setupsim()} \\
%\var{plot(`x',`rho')} \\
%\var{plotanalytical(`x',`rho')} \\
%\var{limit(`x',-1.1,1.1)} \\
%\var{run()} \\
%\var{block()} \\
%\end{tabular} \\

\lstinputlisting{../examples/example07.py}



\noindent For some sets of initial conditions (e.g. shocktubes), the analytical solution is provided in one of the python modules (\var{gandalf.analysis.analytical}).  The python module will read in the values specified in the simulation parameters file and return the correct solution for that given snapshot time.  To overplot a specified quantity (e.g. \var{x} vs density), then use the command \singlecommand{plotanalytical(`x',`rho')}  Another useful command to set the limits of any plot quantity is \singlecommand{limit(quantity,min,max)} where \var{quantity} is a string of the quantity and \var{min} and \var{max} define the range of the plot.  In this example, \lstinline{limit(`x',-1.1,1.1)}
%\newpage


%%%%%%%%%%%%%%%%%%%%%%%%%%%%%%%%%%%%%%%%%%%%%%%%%%%%%%%%%%%%%%%%%%%%%%%%%%%%%%%
\subsubsection{Example 8 - Creating initial conditions directly in the python script}
In this example, we demonstrate how to generate some simple initial conditions inside a python script.  For simplicity, we start with a simple 1D shocktube example. \\

%\begin{tabular}{p{14.0cm}}
%\var{from gandalf.analysis.facade import *} \\
%\var{import numpy as np} \\
%\var{import time} \\
%\var{Nsph = 200} \\
%\var{vfluid = 4.0} \\
%\var{xmin = -1.5} \\
%\var{xmax = 1.5} \\
%\var{deltax = (xmax - xmin) / Nsph} \\
%\var{x = np.linspace(xmin + 0.5*deltax,xmax - 0.5*deltax,num=Nsph)} \\
%\var{m = np.ones(Nsph)*(xmax - xmin)/Nsph} \\
%\var{vx = np.ones(Nsph)*vfluid}  \\
%\var{vx[x > 0] = -vfluid} \\
%\var{sim = newsim(ndim=1,sim=`sph')} \\
%\var{sim.SetParam(`ic',`python')} \\
%\var{sim.SetParam(`gas\_eos',`isothermal')}  \\
%\var{sim.SetParam(`Nsph',200)}  \\
%\var{sim.SetParam(`tend',0.2)}  \\
%\var{sim.SetParam(`dt\_snap',0.05)} \\
%\var{sim.SetParam(`dimensionless',1)} \\
%\var{sim.SetParam(`vfluid1[0]',vfluid)} \\
%\var{sim.SetParam(`vfluid2[0]',-vfluid)} \\
%\var{sim.SetParam(`boxmin[0]',xmin)} \\
%\var{sim.SetParam(`boxmax[0]',xmax)} \\
%\var{sim.SetParam(`run\_id',`SHOCKTUBE1')} \\
%\var{sim.PreSetupForPython()} \\
%\var{sim.ImportArray(x,`x')} \\
%\var{sim.ImportArray(vx,`vx')} \\
%\var{sim.ImportArray(m,`m')}   \\
%\var{sim.SetupSimulation()}  \\
%\var{plot(`x',`rho')} \\
%\var{plotanalytical(`x',`rho',ic=`shocktube')} \\
%\var{limit(`x',-0.17,0.17,window=`all')} \\
%\var{limit(`rho',0.0,21.0,window=`all')} \\
%\var{run()} \\
%\var{savefig(`shocktube.png')} \\
%\var{block()} \\
%\end{tabular} \\

\lstinputlisting{../examples/example08.py}



\noindent As well as importing the GANDALF Python library, we must also import the NUMPY library, which is a popular mathematical library in Python for handling arrays, with the command \singlecommand{import numpy as np}
We first create the initial conditions in Python before exporting the data to the C++ part of the code.  This can be done by creating NUMPY arrays for each variable, e.g. \var{x}, \var{m}, etc..  Note that even for vector quantities (e.g. position, velocity), we must allocate separate 1D NUMPY arrays for each component, for example the commands \\
\newline
\indent \lstinline{a1 = np.zeros(N)}, \\
\indent \lstinline{a2 = np.ones(N)}, \\
\indent \lstinline{a3 = np.linspace(0.0,1.0,num=N)} \\
\newline
return NUMPY arrays of size \var{N} with all zeros, all ones, and uniform values between 0 and 1 respectively  (See NUMPY website and documentation for more information on NUMPY functions).  For this simple shocktube test, we set all velocities left of the shock interface to $4.0$ and right of the shock interface to $-4.0$ with the commands \lstinline{vx = np.ones(Nsph)*4.0; vx[x > 0.0] = -4.0}.

We then create an undefined 1D simulation object with \lstinline{sim = newsim(ndim=1)} and set various parameters determing the simulation type, integration scheme, equation of state, etc..  We must then inform the code that we intend to generate initial conditions in the python script (and not say from an external file or one of the internal C++ routines) by the command \lstinline{sim.SetParam('ic','python')}.  We must also inform the code of the number of SPH particles to be used with the command \lstinline{sim.SetParam('Nsph',Nsph)}, in order to allocate enough memory for the simulation (N.B. \var{Nsph} is simply a local Python variable in this example.  This can be any other variable, or even simply an integer number).

The last step before importing the arrays to C++, we must initialise various things in the code, including importantly allocating memory for the particles, by calling the function \singlecommand{sim.PreSetupforPython()}
Next, we can import the array into the C++ arrays by the command \singlecommand{sim.ImportArray(numpyarray,varname)}
where \var{numpyarray} is the local NUMPY array and \var{varname} is a string containing the C++ variable name.  For example, to import the x-positions, we call the command \lstinline{sim.ImportArray(x,'x')}.  Once all arrays have been called, we can finally call the setup function \lstinline{sim.SetupSimulation()} and plot and run the simulation to completion.



%%%%%%%%%%%%%%%%%%%%%%%%%%%%%%%%%%%%%%%%%%%%%%%%%%%%%%%%%%%%%%%%%%%%%%%%%%%%%%%
\subsubsection{Example  9 - Creating initial conditions for N-body simulation in python script}
In this example, we demonstrate how to generate some simple initial conditions for pure N-body simulation inside a python script. \\

%\begin{tabular}{p{14.0cm}}
%\var{from gandalf.analysis.facade import *} \\
%\var{import numpy as np} \\
%\var{import time} \\
%\var{Nstar = 3} \\
%\var{x = np.zeros(Nstar)} \\
%\var{y = np.zeros(Nstar)} \\
%\var{vx = np.zeros(Nstar)} \\
%\var{vy = np.zeros(Nstar)} \\
%\var{m = np.zeros(Nstar)} \\
%\var{h = 0.000001*np.ones(Nstar)} \\
%\var{m[0] = 3.0;}\,\,\, \var{x[0] = 1.0;}\,\,\,\var{y[0] = 3.0} \\
%\var{m[0] = 4.0;}\,\,\, \var{x[0] = -2.0;}\,\,\,\var{y[0] = -1.0} \\
%\var{m[0] = 5.0;}\,\,\, \var{x[0] = 1.0;}\,\,\,\var{y[0] = -1.0} \\
%\var{sim = newsim(ndim=2,sim=`nbody')} \\
%\var{sim.SetParam(`ic',`python')} \\
%\var{sim.SetParam(`nbody',`hermite4ts')} \\
%\var{sim.SetParam(`sub\_systems',0)}  \\
%\var{sim.SetParam(`Npec',3)} \\
%\var{sim.SetParam(`Nlevels',1)} \\
%\var{sim.SetParam(`tend',80.0)} \\
%\var{sim.SetParam(`dt\_snap',1.0)} \\
%\var{sim.SetParam(`noutputstep',128)} \\
%\var{sim.SetParam(`ndiagstep',2048)} \\
%\var{sim.SetParam(`dimensionless',1)} \\
%\var{sim.SetParam(`run\_id',`BURRAU1')} \\
%\var{sim.PreSetupForPython()} \\
%\var{sim.ImportArray(x,`x',`star')} \\
%\var{sim.ImportArray(y,`y',`star')} \\
%\var{sim.ImportArray(vx,`vx',`star')} \\
%\var{sim.ImportArray(vx,`vy',`star')} \\
%\var{sim.ImportArray(m,`m',`star')}   \\
%\var{sim.ImportArray(h,`h',`star')}   \\
%\var{sim.SetupSimulation()}  \\
%\var{plot(`x',`y',type=`star')} \\
%\var{limit(`x',-30.0,30.0,window=`all')} \\
%\var{limit(`y',-20.0,40.0,window=`all')} \\
%\var{run()} \\
%\var{block()} \\
%\end{tabular} \\

\lstinputlisting{../examples/example09.py}




%%%%%%%%%%%%%%%%%%%%%%%%%%%%%%%%%%%%%%%%%%%%%%%%%%%%%%%%%%%%%%%%%%%%%%%%%%%%%%%
\subsubsection{Example  10 - Generating rendered images from SPH simulations}
In this example, we show how to generate rendered images of some quantity.  Note that rendered plots can only be created for simulations using 2 or 3 dimensions. \\

%\begin{tabular}{p{14.0cm}}
%\var{from gandalf.analysis.facade import *} \\
%\var{sim = newsim(`khi.dat')} \\
%\var{setupsim()} \\
%\var{render(`x',`y','rho',res=128)} \\
%\var{limit(`x',-0.5,0.5)} \\
%\var{limit(`y',-0.5,0.5)} \\
%\var{limit(`rho',1.0,2.0)} \\
%\var{run()} \\
%\var{block()} \\
%\end{tabular} \\

\lstinputlisting{../examples/example10.py}



\noindent A rendered plot of the density on the x-y plane can be created using the command \singlecommand{render(`x',`y',`rho')}  The first argument specifies the x-coordinate, the second argument the y-coordinate and the third argument the rendered quantity.  An important optional argument is \var{res} which specifies the resolution (in pixels) of the rendered image in each dimension.  If the image is not square, then the user may pass the resolution in each dimension in parenthesis, e.g. \lstinline{render(`x',`y','rho',res=(128,64))}.


%%%%%%%%%%%%%%%%%%%%%%%%%%%%%%%%%%%%%%%%%%%%%%%%%%%%%%%%%%%%%%%%%%%%%%%%%%%%%%%
\subsubsection{Example  11 - Plotting in alternative coordinate systems}
In this example, we show how to plot in alternative coordinate systems (i.e. other than Cartesian coordinates). \\

%\begin{tabular}{p{14.0cm}}
%\var{from gandalf.analysis.facade import *} \\
%\var{sim = loadsim(`SEDOV1')} \\
%\var{plot(`r',`rho')} \\
%\var{block()} \\
%\end{tabular} \\

\lstinputlisting{../examples/example11.py}




\noindent Once the simulation has been loaded or set-up as usual, then it is possible to plot in 3 pre-defined coordinate systems, Cartesian, spherical polar and cylindrical polar coordinates, or even a mixture.  Cartesian coordinates are represented by the strings, \var{x}, \var{y}, \var{z}; sphericial polar coordinates by the strings \var{r}, \var{theta}, \var{phi}; cylindrical coordinates by the strings \var{R}, \var{phi}, \var{z}.  For example, to plot the density as a function of radius (for spherically symmetric simulations), then we can use the command \singlecommand{plot(`r',`rho')}



%%%%%%%%%%%%%%%%%%%%%%%%%%%%%%%%%%%%%%%%%%%%%%%%%%%%%%%%%%%%%%%%%%%%%%%%%%%%%%%
\subsubsection{Example  12 - Changing the plotting units}
In this example, we show how to plot using different units to those used in the simulation or provided in the snapshot files. \\

%\begin{tabular}{p{14.0cm}}
%\var{from gandalf.analysis.facade import *} \\
%\var{sim = newsim(`bossbodenheimer.dat')} \\
%\var{sim.SetParam(`tend',0.02)} \\
%\var{setupsim()} \\
%\var{plot(`x',`y')} \\
%\var{run()} \\
%\var{block()} \\
%\var{plot(`x',`y',xunit='au',yunit='au')} \\
%\end{tabular} \\

\lstinputlisting{../examples/example12.py}


\noindent In order to plot quantities using a different unit to the default, the optional arguments \var{xunit} and \var{yunit} must be appended to the argument list for the relevant plotting command.  These arguments contain strings of the required unit (See section \ref{SS:UNITPARAMS} for the list of available units).


%%%%%%%%%%%%%%%%%%%%%%%%%%%%%%%%%%%%%%%%%%%%%%%%%%%%%%%%%%%%%%%%%%%%%%%%%%%%%%%
\subsubsection{Example  13 - Creating and plotting user-defined quantities}
In this example, we show how to create new formulaic quantities to be plotted as regular quantities by the python library. \\

%\begin{tabular}{p{14.0cm}}
%\var{from gandalf.analysis.facade import *} \\
%\var{sim = loadsim(`TEST1')} \\
%\var{CreateUserQuantity(`ke',`0.5*m*vx*vx',scaling\_factor=`u',label=`frac\{1\}\{2\}mv\^2')} \\
%\var{plot(`x',`ke',snap=2)} \\
%\var{limit(`ke',0.0,2.7)} \\
%\var{addplot(`x',`u',snap=2)} \\
%\var{block()} \\
%\end{tabular} \\

\lstinputlisting{../examples/example13.py}



\noindent We can create new quantities from existing particle properties using the command \singlecommand{CreateUserQuantity(quantity\_name,quantity\_formula)} where \var{quantity\_name} is the new string name of the derived quantity and \var{quantity\_formula} is string containing an algebraic formula that defines the new quantity.  In our example, we create a new quantity to compute the kinetic energy of a particle with the algebraic formula \var{0.5*m*vx*vx}.  For 2D and 3D simulations, this would become \var{0.5*m*(vx*vx + vy*vy)} and \var{0.5*m*(vx*vx + vy*vy + vz*vz)} respectively.  Note that the algebraic formula can also use parameters by using the same string as defined in the parameters file.

The new quantity can then be plotted as any other regular quantity, e.g. \var{plot(`x',`ke')}.  Two important optional arguments are \var{scaling\_factor}, which defines the string of the unit (e.g. in this case \var{E} since \var{ke} is an energy), and \var{label}, which is a latex string used on the plots (e.g. \var{frac\{1\}\{2\}mv\^2}).



%%%%%%%%%%%%%%%%%%%%%%%%%%%%%%%%%%%%%%%%%%%%%%%%%%%%%%%%%%%%%%%%%%%%%%%%%%%%%%%
\subsubsection{Example  14 - Plotting time series of particle properties}
In this example, we show how to plot a time series of particle quantities from each snapshot recorded in the simulation buffer.  This can be used to plot useful quantities versus time (e.g. mass of a sink versus time), or x-y tracks of a particle as it moves in space. \\

%\begin{tabular}{p{14.0cm}}
%\var{from gandalf.analysis.facade import *} \\
%\var{sim = loadsim(`TEST1')} \\
%\var{time\_plot(`t',`rho',id=0,linestyle=`-')} \\
%\var{block()} \\
%\var{time\_plot(`x',`y',id=0,linestyle=`-')} \\
%\var{block()} \\
%\end{tabular} \\

\lstinputlisting{../examples/example14.py}



\noindent In order to plot a time series of some particle property, we use the command \singlecommand{time\_plot(`t',`rho',id=0)} where \var{id} is the unique/original id of the particle to be plotted.  If no id is given, then an error is returned and nothing will be plotted on screen.



%%%%%%%%%%%%%%%%%%%%%%%%%%%%%%%%%%%%%%%%%%%%%%%%%%%%%%%%%%%%%%%%%%%%%%%%%%%%%%%
%\subsubsection{Example  15 - Calling analysis routines and plotting results}
%To be done



%%%%%%%%%%%%%%%%%%%%%%%%%%%%%%%%%%%%%%%%%%%%%%%%%%%%%%%%%%%%%%%%%%%%%%%%%%%%%%%
%\subsubsection{Example  16 - Creating and plotting multiple sub-plots on a single window}
%To be done



%%%%%%%%%%%%%%%%%%%%%%%%%%%%%%%%%%%%%%%%%%%%%%%%%%%%%%%%%%%%%%%%%%%%%%%%%%%%%%%
\subsubsection{Example  17 - Creating an animation from simulation snapshot files}
%In this example, we show how to generate a movie from a series of snapshot files loaded into memory. \\


\begin{tabular}{p{14.0cm}}
\var{from gandalf.analysis.facade import *} \\
\var{sim = loadsim(`TEST1')} \\
\var{plot(`x',`y')} \\
\var{make\_movie(`MOVIE1.mp4')} \\
\var{break()} \\
\end{tabular} \\
\newline
%\lstinputlisting{../examples/example17.py}




\noindent Before generating a movie, we must first specify the plot commands that are to be used to generate the movie.  For example, in order to show how the particles move on the x-y plane in the movie, we must first plot this with the regular plot command, \lstinline{plot('x','y')}.  Once the commands to generate the plot window have been completed, we then generate the movie with the command \singlecommand{make\_movie(`MOVIE1.mp4')} where the only required argument is the filename of the movie.  The python library will then step through each snapshot in turn, generate the image as a temporary png file, and then attach all images together using ffmpeg.  All temporary png files should then be deleted automatically from the disc.


\newpage


%%%%%%%%%%%%%%%%%%%%%%%%%%%%%%%%%%%%%%%%%%%%%%%%%%%%%%%%%%%%%%%%%%%%%%%%%%%%%%%
\section{Developer notes}
GANDALF makes heavy use of the object-oriented features in C++ in order to write a flexible code that can used in many different ways while maintaining good performance.  For developers that wish to modify sections of GANDALF, or add new classes or functions, it is recommended to read this section in order to understand some of the ...


%%%%%%%%%%%%%%%%%%%%%%%%%%%%%%%%%%%%%%%%%%%%%%%%%%%%%%%%%%%%%%%%%%%%%%%%%%%%%%%
\subsection{Class hieararchy}
GANDALF uses a hierarchy of classes using inheritance to structure the code in a logical and maintainable manner.  Although classes permeate the code at almost all levels, we briefly discuss the main classes here.

\subsubsection{Simulation class}
The \var{Simulation} class is the class that contains every other class, including SPH and N-body classes, in order to fully run the simulation.  The \var{Simulation} class is split into three levels :
\begin{itemize}
\item \var{SimulationBase} - Non-templated base class for generating the simulation class.  Provides binding layer between C++ code and Python (since Python cannot link to templated classes).  Contains all subroutines that can be called by Python.
\item \var{Simulation} - Main parent simulation class that contains all common functions for all possible simulations.  Inherits from \var{SimulationBase}.
\item \var{AlgorithmSimulation} - Child class containing specific algorithmic implementation.  For example, `grad-h' SPH simulations are run by the \var{GradhSphSimulation} class; pure N-body simulations are run by the \var{NbodySimulation} class, etc..
\end{itemize}

\subsubsection{Sph class}

\subsubsection{SphKernel class}

\subsubsection{SphIntegration class}

\subsubsection{SphNeighbourSearch class}

\subsubsection{EOS class}

\subsubsection{Nbody class}

\subsubsection{Sinks class}



%%%%%%%%%%%%%%%%%%%%%%%%%%%%%%%%%%%%%%%%%%%%%%%%%%%%%%%%%%%%%%%%%%%%%%%%%%%%%%%
\subsection{Templates}
Templates are used in C++ to write generic code that can be used by different data types throughout the code without needing to write multiple versions of essentially the same code.  It can also be used as an optimisation technique for performance, which we explain later.

The simplest case of using templates in GANDALF is for dimensionality.  For almost all of the code, we use the template parameter `\var{ndim}' to generate a version of the code for each dimensionality used, i.e. $1$, $2$ and $3$.  In principle, \var{ndim} could simply be a variable which is defined by the user depending on which dimensionality is used for that simulation.  However, the value of \var{ndim} would then need to be accessed in memory very frequently.  By defining \var{ndim} as a template parameter, \var{ndim} will be defined statically and therefore does not require the overhead of accessing the memory address each time.

Throughout the code, the variables that are used as template parameters are :
\begin{itemize}
\item \var{ndim} : Number of dimensions
\item \var{SphKernel} : SPH kernel function used
\item \var{ParticleType} : Particle data structure for employed hydrodynamics method
\item \var{CellType} : Radiation transport tree cell-type, depending on employed RT method
\end{itemize}


%%%%%%%%%%%%%%%%%%%%%%%%%%%%%%%%%%%%%%%%%%%%%%%%%%%%%%%%%%%%%%%%%%%%%%%%%%%%%%%
\subsection{Particle data structures}

To optimise the speed of the code, GANDALF uses data structures that contain only the quantities for that particular hydrodynamics algorithm.  For example, `grad-h' SPH in 1D requires a very different number of variables to Godunov SPH in 3D.  Therefore, a different version of each sub-routine that uses the SPH particle data directly is required using templates.

The various particle data structures are defined in the header file \var{`SphParticle.h'}.  All particles are derived from a base data-type called \var{`Particle}

%Need a figure showing the particle data structure inheritance hierarchy here in the future.



%%%%%%%%%%%%%%%%%%%%%%%%%%%%%%%%%%%%%%%%%%%%%%%%%%%%%%%%%%%%%%%%%%%%%%%%%%%%%%%
\subsection{Particle array pointers}

If a class which does not know the exact particle data structure wishes to access an array, or a single array element.



\newpage


%%%%%%%%%%%%%%%%%%%%%%%%%%%%%%%%%%%%%%%%%%%%%%%%%%%%%%%%%%%%%%%%%%%%%%%%%%%%%%%
\section{To-do list}

\subsection{Known bugs}
List of known bugs as of version \VERNO.

\begin{itemize}
\item Godunov SPH and Saitoh \& Makino (2012) SPH have not been fully updated since recent code refactoring.
\item For very short simulations, plots may not be updated correctly since the simulation process finishes before the plotting process has recieved its commands.
\item If running a simulation in interactive mode and a different simulation is loaded into memory, then it is no longer possible to continue running that simulation.
\item Rendered images are technically not done correct if smoothing lengths are smaller than the grid size (which can often be the case, although the images are fine for viewing/movie purposes).
\item Rendered images do not correctly calculate automatic limits; must be inserted manually using the limit command.
\end{itemize}


\subsection{Proposed features}
List of possible new features for future versions.

\begin{itemize}
\item Finish implementation of MPI
\item Allow ability to add titles to graphs in matplotlib
\item Include analytical solutions for Sedov blast-wave test and freefall collapse test
\item Plot star/sink-based statistics, e.g. sink mass-functions, binary statistics
\item Include analysis routines for python for N-body specific statistics, e.g. Q-parameter, $\lambda$-parameter, etc..
\item Strict energy-error checking for N-body simulations
\item More sanity-checking, error-trapping and assert statements (to prevent crashing on erroneous input and to help debugging purposes).
\item Update Godunov SPH to 2nd-order using slope limiters.
\end{itemize}


\end{document}
%%%%%%%%%%%%%%%%%%%%%%%%%%%%%%%%%%%%%%%%%%%%%%%%%%%%%%%%%%%%%%%%%%%%%%%%%%%%%%
