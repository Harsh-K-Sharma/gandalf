% Generated by Sphinx.
\def\sphinxdocclass{report}
\documentclass[letterpaper,10pt,english]{sphinxmanual}
\usepackage[utf8]{inputenc}
\DeclareUnicodeCharacter{00A0}{\nobreakspace}
\usepackage{cmap}
\usepackage[T1]{fontenc}
\usepackage{babel}
\usepackage{times}
\usepackage[Bjarne]{fncychap}
\usepackage{longtable}
\usepackage{sphinx}
\usepackage{multirow}

\addto\captionsenglish{\renewcommand{\figurename}{Fig. }}
\addto\captionsenglish{\renewcommand{\tablename}{Table }}
\floatname{literal-block}{Listing }



\title{GANDALF Documentation}
\date{November 09, 2015}
\release{0.4.0}
\author{David Hubber; Giovanni Rosotti}
\newcommand{\sphinxlogo}{}
\renewcommand{\releasename}{Release}
\makeindex

\makeatletter
\def\PYG@reset{\let\PYG@it=\relax \let\PYG@bf=\relax%
    \let\PYG@ul=\relax \let\PYG@tc=\relax%
    \let\PYG@bc=\relax \let\PYG@ff=\relax}
\def\PYG@tok#1{\csname PYG@tok@#1\endcsname}
\def\PYG@toks#1+{\ifx\relax#1\empty\else%
    \PYG@tok{#1}\expandafter\PYG@toks\fi}
\def\PYG@do#1{\PYG@bc{\PYG@tc{\PYG@ul{%
    \PYG@it{\PYG@bf{\PYG@ff{#1}}}}}}}
\def\PYG#1#2{\PYG@reset\PYG@toks#1+\relax+\PYG@do{#2}}

\expandafter\def\csname PYG@tok@gd\endcsname{\def\PYG@tc##1{\textcolor[rgb]{0.63,0.00,0.00}{##1}}}
\expandafter\def\csname PYG@tok@gu\endcsname{\let\PYG@bf=\textbf\def\PYG@tc##1{\textcolor[rgb]{0.50,0.00,0.50}{##1}}}
\expandafter\def\csname PYG@tok@gt\endcsname{\def\PYG@tc##1{\textcolor[rgb]{0.00,0.27,0.87}{##1}}}
\expandafter\def\csname PYG@tok@gs\endcsname{\let\PYG@bf=\textbf}
\expandafter\def\csname PYG@tok@gr\endcsname{\def\PYG@tc##1{\textcolor[rgb]{1.00,0.00,0.00}{##1}}}
\expandafter\def\csname PYG@tok@cm\endcsname{\let\PYG@it=\textit\def\PYG@tc##1{\textcolor[rgb]{0.25,0.50,0.56}{##1}}}
\expandafter\def\csname PYG@tok@vg\endcsname{\def\PYG@tc##1{\textcolor[rgb]{0.73,0.38,0.84}{##1}}}
\expandafter\def\csname PYG@tok@m\endcsname{\def\PYG@tc##1{\textcolor[rgb]{0.13,0.50,0.31}{##1}}}
\expandafter\def\csname PYG@tok@mh\endcsname{\def\PYG@tc##1{\textcolor[rgb]{0.13,0.50,0.31}{##1}}}
\expandafter\def\csname PYG@tok@cs\endcsname{\def\PYG@tc##1{\textcolor[rgb]{0.25,0.50,0.56}{##1}}\def\PYG@bc##1{\setlength{\fboxsep}{0pt}\colorbox[rgb]{1.00,0.94,0.94}{\strut ##1}}}
\expandafter\def\csname PYG@tok@ge\endcsname{\let\PYG@it=\textit}
\expandafter\def\csname PYG@tok@vc\endcsname{\def\PYG@tc##1{\textcolor[rgb]{0.73,0.38,0.84}{##1}}}
\expandafter\def\csname PYG@tok@il\endcsname{\def\PYG@tc##1{\textcolor[rgb]{0.13,0.50,0.31}{##1}}}
\expandafter\def\csname PYG@tok@go\endcsname{\def\PYG@tc##1{\textcolor[rgb]{0.20,0.20,0.20}{##1}}}
\expandafter\def\csname PYG@tok@cp\endcsname{\def\PYG@tc##1{\textcolor[rgb]{0.00,0.44,0.13}{##1}}}
\expandafter\def\csname PYG@tok@gi\endcsname{\def\PYG@tc##1{\textcolor[rgb]{0.00,0.63,0.00}{##1}}}
\expandafter\def\csname PYG@tok@gh\endcsname{\let\PYG@bf=\textbf\def\PYG@tc##1{\textcolor[rgb]{0.00,0.00,0.50}{##1}}}
\expandafter\def\csname PYG@tok@ni\endcsname{\let\PYG@bf=\textbf\def\PYG@tc##1{\textcolor[rgb]{0.84,0.33,0.22}{##1}}}
\expandafter\def\csname PYG@tok@nl\endcsname{\let\PYG@bf=\textbf\def\PYG@tc##1{\textcolor[rgb]{0.00,0.13,0.44}{##1}}}
\expandafter\def\csname PYG@tok@nn\endcsname{\let\PYG@bf=\textbf\def\PYG@tc##1{\textcolor[rgb]{0.05,0.52,0.71}{##1}}}
\expandafter\def\csname PYG@tok@no\endcsname{\def\PYG@tc##1{\textcolor[rgb]{0.38,0.68,0.84}{##1}}}
\expandafter\def\csname PYG@tok@na\endcsname{\def\PYG@tc##1{\textcolor[rgb]{0.25,0.44,0.63}{##1}}}
\expandafter\def\csname PYG@tok@nb\endcsname{\def\PYG@tc##1{\textcolor[rgb]{0.00,0.44,0.13}{##1}}}
\expandafter\def\csname PYG@tok@nc\endcsname{\let\PYG@bf=\textbf\def\PYG@tc##1{\textcolor[rgb]{0.05,0.52,0.71}{##1}}}
\expandafter\def\csname PYG@tok@nd\endcsname{\let\PYG@bf=\textbf\def\PYG@tc##1{\textcolor[rgb]{0.33,0.33,0.33}{##1}}}
\expandafter\def\csname PYG@tok@ne\endcsname{\def\PYG@tc##1{\textcolor[rgb]{0.00,0.44,0.13}{##1}}}
\expandafter\def\csname PYG@tok@nf\endcsname{\def\PYG@tc##1{\textcolor[rgb]{0.02,0.16,0.49}{##1}}}
\expandafter\def\csname PYG@tok@si\endcsname{\let\PYG@it=\textit\def\PYG@tc##1{\textcolor[rgb]{0.44,0.63,0.82}{##1}}}
\expandafter\def\csname PYG@tok@s2\endcsname{\def\PYG@tc##1{\textcolor[rgb]{0.25,0.44,0.63}{##1}}}
\expandafter\def\csname PYG@tok@vi\endcsname{\def\PYG@tc##1{\textcolor[rgb]{0.73,0.38,0.84}{##1}}}
\expandafter\def\csname PYG@tok@nt\endcsname{\let\PYG@bf=\textbf\def\PYG@tc##1{\textcolor[rgb]{0.02,0.16,0.45}{##1}}}
\expandafter\def\csname PYG@tok@nv\endcsname{\def\PYG@tc##1{\textcolor[rgb]{0.73,0.38,0.84}{##1}}}
\expandafter\def\csname PYG@tok@s1\endcsname{\def\PYG@tc##1{\textcolor[rgb]{0.25,0.44,0.63}{##1}}}
\expandafter\def\csname PYG@tok@gp\endcsname{\let\PYG@bf=\textbf\def\PYG@tc##1{\textcolor[rgb]{0.78,0.36,0.04}{##1}}}
\expandafter\def\csname PYG@tok@sh\endcsname{\def\PYG@tc##1{\textcolor[rgb]{0.25,0.44,0.63}{##1}}}
\expandafter\def\csname PYG@tok@ow\endcsname{\let\PYG@bf=\textbf\def\PYG@tc##1{\textcolor[rgb]{0.00,0.44,0.13}{##1}}}
\expandafter\def\csname PYG@tok@sx\endcsname{\def\PYG@tc##1{\textcolor[rgb]{0.78,0.36,0.04}{##1}}}
\expandafter\def\csname PYG@tok@bp\endcsname{\def\PYG@tc##1{\textcolor[rgb]{0.00,0.44,0.13}{##1}}}
\expandafter\def\csname PYG@tok@c1\endcsname{\let\PYG@it=\textit\def\PYG@tc##1{\textcolor[rgb]{0.25,0.50,0.56}{##1}}}
\expandafter\def\csname PYG@tok@kc\endcsname{\let\PYG@bf=\textbf\def\PYG@tc##1{\textcolor[rgb]{0.00,0.44,0.13}{##1}}}
\expandafter\def\csname PYG@tok@c\endcsname{\let\PYG@it=\textit\def\PYG@tc##1{\textcolor[rgb]{0.25,0.50,0.56}{##1}}}
\expandafter\def\csname PYG@tok@mf\endcsname{\def\PYG@tc##1{\textcolor[rgb]{0.13,0.50,0.31}{##1}}}
\expandafter\def\csname PYG@tok@err\endcsname{\def\PYG@bc##1{\setlength{\fboxsep}{0pt}\fcolorbox[rgb]{1.00,0.00,0.00}{1,1,1}{\strut ##1}}}
\expandafter\def\csname PYG@tok@mb\endcsname{\def\PYG@tc##1{\textcolor[rgb]{0.13,0.50,0.31}{##1}}}
\expandafter\def\csname PYG@tok@ss\endcsname{\def\PYG@tc##1{\textcolor[rgb]{0.32,0.47,0.09}{##1}}}
\expandafter\def\csname PYG@tok@sr\endcsname{\def\PYG@tc##1{\textcolor[rgb]{0.14,0.33,0.53}{##1}}}
\expandafter\def\csname PYG@tok@mo\endcsname{\def\PYG@tc##1{\textcolor[rgb]{0.13,0.50,0.31}{##1}}}
\expandafter\def\csname PYG@tok@kd\endcsname{\let\PYG@bf=\textbf\def\PYG@tc##1{\textcolor[rgb]{0.00,0.44,0.13}{##1}}}
\expandafter\def\csname PYG@tok@mi\endcsname{\def\PYG@tc##1{\textcolor[rgb]{0.13,0.50,0.31}{##1}}}
\expandafter\def\csname PYG@tok@kn\endcsname{\let\PYG@bf=\textbf\def\PYG@tc##1{\textcolor[rgb]{0.00,0.44,0.13}{##1}}}
\expandafter\def\csname PYG@tok@o\endcsname{\def\PYG@tc##1{\textcolor[rgb]{0.40,0.40,0.40}{##1}}}
\expandafter\def\csname PYG@tok@kr\endcsname{\let\PYG@bf=\textbf\def\PYG@tc##1{\textcolor[rgb]{0.00,0.44,0.13}{##1}}}
\expandafter\def\csname PYG@tok@s\endcsname{\def\PYG@tc##1{\textcolor[rgb]{0.25,0.44,0.63}{##1}}}
\expandafter\def\csname PYG@tok@kp\endcsname{\def\PYG@tc##1{\textcolor[rgb]{0.00,0.44,0.13}{##1}}}
\expandafter\def\csname PYG@tok@w\endcsname{\def\PYG@tc##1{\textcolor[rgb]{0.73,0.73,0.73}{##1}}}
\expandafter\def\csname PYG@tok@kt\endcsname{\def\PYG@tc##1{\textcolor[rgb]{0.56,0.13,0.00}{##1}}}
\expandafter\def\csname PYG@tok@sc\endcsname{\def\PYG@tc##1{\textcolor[rgb]{0.25,0.44,0.63}{##1}}}
\expandafter\def\csname PYG@tok@sb\endcsname{\def\PYG@tc##1{\textcolor[rgb]{0.25,0.44,0.63}{##1}}}
\expandafter\def\csname PYG@tok@k\endcsname{\let\PYG@bf=\textbf\def\PYG@tc##1{\textcolor[rgb]{0.00,0.44,0.13}{##1}}}
\expandafter\def\csname PYG@tok@se\endcsname{\let\PYG@bf=\textbf\def\PYG@tc##1{\textcolor[rgb]{0.25,0.44,0.63}{##1}}}
\expandafter\def\csname PYG@tok@sd\endcsname{\let\PYG@it=\textit\def\PYG@tc##1{\textcolor[rgb]{0.25,0.44,0.63}{##1}}}

\def\PYGZbs{\char`\\}
\def\PYGZus{\char`\_}
\def\PYGZob{\char`\{}
\def\PYGZcb{\char`\}}
\def\PYGZca{\char`\^}
\def\PYGZam{\char`\&}
\def\PYGZlt{\char`\<}
\def\PYGZgt{\char`\>}
\def\PYGZsh{\char`\#}
\def\PYGZpc{\char`\%}
\def\PYGZdl{\char`\$}
\def\PYGZhy{\char`\-}
\def\PYGZsq{\char`\'}
\def\PYGZdq{\char`\"}
\def\PYGZti{\char`\~}
% for compatibility with earlier versions
\def\PYGZat{@}
\def\PYGZlb{[}
\def\PYGZrb{]}
\makeatother

\renewcommand\PYGZsq{\textquotesingle}

\begin{document}

\maketitle
\tableofcontents
\phantomsection\label{index::doc}\phantomsection\label{index:module-facade}\index{facade (module)}\index{ListFunctions() (in module facade)}

\begin{fulllineitems}
\phantomsection\label{index:facade.ListFunctions}\pysiglinewithargsret{\code{facade.}\bfcode{ListFunctions}}{}{}
List the available functions defined in facade

\end{fulllineitems}

\index{newsim() (in module facade)}

\begin{fulllineitems}
\phantomsection\label{index:facade.newsim}\pysiglinewithargsret{\code{facade.}\bfcode{newsim}}{\emph{paramfile=None}, \emph{ndim=None}, \emph{sim=None}}{}
Create a new simulation object. Need to specify either the parameter
file, or the number of dimensions and the simulation type. Note that it is not
possible to change the number of dimensions afterwards or simulation type
afterwards.

\end{fulllineitems}

\index{loadsim() (in module facade)}

\begin{fulllineitems}
\phantomsection\label{index:facade.loadsim}\pysiglinewithargsret{\code{facade.}\bfcode{loadsim}}{\emph{run\_id}, \emph{fileformat=None}, \emph{buffer\_flag='cache'}}{}
Given the run\_id of a simulation, reads it from the disk.
Returns the newly created simulation object.
\begin{description}
\item[{Required arguments:}] \leavevmode
run\_id      : Simulation run identification string.

\item[{Optional arguments:}] \leavevmode
fileformat  : Format of all snapshot files of simulation.
buffer\_flag : Record snapshot data in simulation buffer.

\end{description}

\end{fulllineitems}

\index{run() (in module facade)}

\begin{fulllineitems}
\phantomsection\label{index:facade.run}\pysiglinewithargsret{\code{facade.}\bfcode{run}}{\emph{no=None}}{}
Run a simulation. If no argument is given, run the current one;
otherwise queries the buffer for the given simulation number.
If the simulation has not been setup, does it before running.
\begin{description}
\item[{Optional arguments:}] \leavevmode
no         : Simulation number

\end{description}

\end{fulllineitems}

\index{plot() (in module facade)}

\begin{fulllineitems}
\phantomsection\label{index:facade.plot}\pysiglinewithargsret{\code{facade.}\bfcode{plot}}{\emph{x}, \emph{y}, \emph{type='default'}, \emph{snap='current'}, \emph{sim='current'}, \emph{overplot=False}, \emph{autoscale=False}, \emph{xunit='default'}, \emph{yunit='default'}, \emph{xaxis='linear'}, \emph{yaxis='linear'}, \emph{**kwargs}}{}
Plot particle data as a scatter plot.  Creates a new plotting window if
one does not already exist.
\begin{description}
\item[{Required arguments:}] \leavevmode
x          : Quantity on the x-axis. Must be a string.
y          : Quantity on the y-axis. Must be a string.

\item[{Optional arguments:}] \leavevmode
type       : The type of the particles to plot (e.g. `star' or `sph').
snap       : Number of the snapshot to plot. Defaults to `current'.
sim        : Number of the simulation to plot. Defaults to `current'.
overplot   : If True, overplots on the previous existing plot rather
\begin{quote}

than deleting it. Defaults to False.
\end{quote}
\begin{description}
\item[{autoscale}] \leavevmode{[}If True, the limits of the plot are set{]}
automatically.  Can also be set to `x' or `y' to specify
that only one of the axis has to use autoscaling.
If False (default), autoscaling is not used. On an axis that does
not have autoscaling turned on, global limits are used
if defined for the plotted quantity.

\item[{xunit}] \leavevmode{[}Specify the unit to use for the plotting for the quantity{]}
on the x-axis.

\item[{yunit}] \leavevmode{[}Specify the unit to use for the plotting for the quantity{]}
on the y-axis.

\end{description}

{\color{red}\bfseries{}**}kwargs   : Extra keyword arguments will be passed to matplotlib.

\end{description}

\end{fulllineitems}

\index{addplot() (in module facade)}

\begin{fulllineitems}
\phantomsection\label{index:facade.addplot}\pysiglinewithargsret{\code{facade.}\bfcode{addplot}}{\emph{x}, \emph{y}, \emph{**kwargs}}{}
Thin wrapper around plot that sets overplot to True.  All the other
arguments are the same. If autoscale is not explicitly set, it will be set
to False to preserve the existing settings.
\begin{description}
\item[{Required arguments:}] \leavevmode
x          : Quantity on the x-axis. Must be a string.
y          : Quantity on the y-axis. Must be a string.

\item[{Optional arguments:}] \leavevmode
See plot function optional arguments

\end{description}

\end{fulllineitems}

\index{render() (in module facade)}

\begin{fulllineitems}
\phantomsection\label{index:facade.render}\pysiglinewithargsret{\code{facade.}\bfcode{render}}{\emph{x}, \emph{y}, \emph{render}, \emph{snap='current'}, \emph{sim='current'}, \emph{overplot=False}, \emph{autoscale=False}, \emph{autoscalerender=False}, \emph{coordlimits=None}, \emph{zslice=None}, \emph{xunit='default'}, \emph{yunit='default'}, \emph{renderunit='default'}, \emph{res=64}, \emph{interpolation='nearest'}, \emph{lognorm=False}, \emph{**kwargs}}{}
Create a rendered plot from selected particle data.
\begin{description}
\item[{Required arguments:}] \leavevmode
x          : Quantity on the x-axis. Must be a string.
y          : Quantity on the y-axis. Must be a string.
renderdata : Quantity to be rendered. Must be a string.

\item[{Optional arguments:}] \leavevmode
snap       : Number of the snapshot to plot. Defaults to `current'.
sim        : Number of the simulation to plot. Defaults to `current'.
overplot   : If True, overplots on the previous existing plot rather
\begin{quote}

than deleting it. Defaults to False.
\end{quote}
\begin{description}
\item[{autoscale}] \leavevmode{[}If True, the coordinate limits of the plot are set{]}
automatically.  Can also be set to `x' or `y' to specify
that only one of the axis has to use autoscaling.
If False (default), autoscaling is not used. On an axis that
does not have autoscaling turned on, global limits are used
if defined for the plotted quantity.

\end{description}

autoscalerender : Same as the autoscale, but for the rendered quantity.
coordlimits : Specify the coordinate limits for the plot. In order of
\begin{quote}

precedence, the limits are set in this way:
- What this argument specifies. The value must be an
\begin{quote}

iterable of 4 elements: (xmin, xmax, ymin, ymax).
\end{quote}
\begin{itemize}
\item {} 
If this argument is None (default), global settings for
the quantity are used.

\item {} 
If global settings for the quantity are not defined,
the min and max of the data are used.

\end{itemize}
\end{quote}
\begin{description}
\item[{zslice}] \leavevmode{[}z-coordinate of the slice when doing a slice rendering.{]}
Default is None, which produces a column-integrated plot.
If you set this variable, instead a slice rendering will
be done.

\item[{xunit}] \leavevmode{[}Specify the unit to use for the plotting for the quantity{]}
on the x-axis.

\item[{yunit}] \leavevmode{[}Specify the unit to use for the plotting for the quantity{]}
on the y-axis.

\item[{renderunit}] \leavevmode{[}Specify the unit to use for the plotting for the rendered{]}
quantity.

\item[{res}] \leavevmode{[}Specify the resolution. Can be an integer number, in which{]}
case the same resolution will be used on the two axes, or a
tuple (e.g., (xres, yres)) of two integer numbers, if you
want to specify different resolutions on the two axes.

\item[{interpolation}] \leavevmode{[}Specify the interpolation to use. Default is nearest,{]}
which will show the pixels of the rendering grid. If one
wants to smooth the image, bilinear or bicubic could be
used. See pyplot documentation for the full list of
possible values.

\item[{lognorm}] \leavevmode{[}Boolean flag specifying wheter the colour scale should be{]}
logarithmic (default: linear). If you want to customise the
limits, use the vmin and vmax flags which are passed to
matplotlib

\end{description}

\end{description}

\end{fulllineitems}

\index{addrender() (in module facade)}

\begin{fulllineitems}
\phantomsection\label{index:facade.addrender}\pysiglinewithargsret{\code{facade.}\bfcode{addrender}}{\emph{x}, \emph{y}, \emph{renderq}, \emph{**kwargs}}{}
Thin wrapper around render that sets overplot to True.  If autoscale is
not explicitly set, it will be set to False to preserve the existing settings.
\begin{description}
\item[{Required arguments:}] \leavevmode
x          : Quantity on the x-axis. Must be a string.
y          : Quantity on the y-axis. Must be a string.
renderdata : Quantity to be rendered. Must be a string.

\item[{Optional arguments:}] \leavevmode
See render function optional arguments

\end{description}

\end{fulllineitems}

\index{renderslice() (in module facade)}

\begin{fulllineitems}
\phantomsection\label{index:facade.renderslice}\pysiglinewithargsret{\code{facade.}\bfcode{renderslice}}{\emph{x}, \emph{y}, \emph{renderq}, \emph{zslice}, \emph{**kwargs}}{}
Thin wrapper around render that does slice rendering.
\begin{description}
\item[{Required arguments:}] \leavevmode
x          : Quantity on the x-axis. Must be a string.
y          : Quantity on the y-axis. Must be a string.
renderq    : Quantity to be rendered. Must be a string.
zslice     : z-coordinate of the slice.

\item[{Optional arguments:}] \leavevmode
See render function optional arguments

\end{description}

\end{fulllineitems}

\index{addrenderslice() (in module facade)}

\begin{fulllineitems}
\phantomsection\label{index:facade.addrenderslice}\pysiglinewithargsret{\code{facade.}\bfcode{addrenderslice}}{\emph{x}, \emph{y}, \emph{renderq}, \emph{zslice}, \emph{**kwargs}}{}
Thin wrapper around renderslice that sets overplot to True.  If autoscale is
not explicitly set, it will be set to False to preserve the existing settings.
\begin{description}
\item[{Required arguments:}] \leavevmode
x          : Quantity on the x-axis. Must be a string.
y          : Quantity on the y-axis. Must be a string.
renderq    : Quantity to be rendered. Must be a string.
zslice     : z-coordinate of the slice.

\item[{Optional arguments:}] \leavevmode
See render function optional arguments

\end{description}

\end{fulllineitems}

\index{plotanalytical() (in module facade)}

\begin{fulllineitems}
\phantomsection\label{index:facade.plotanalytical}\pysiglinewithargsret{\code{facade.}\bfcode{plotanalytical}}{\emph{x=None}, \emph{y=None}, \emph{ic='default'}, \emph{snap='current'}, \emph{sim='current'}, \emph{overplot=True}, \emph{autoscale=False}, \emph{xunit='default'}, \emph{yunit='default'}, \emph{time='snaptime'}}{}
Plots the analytical solution.  Reads the problem type from the `ic'
parameter and plots the appropriate solution if implemented.  If no solution
exists, then nothing is plotted.
\begin{description}
\item[{Optional arguments:}] \leavevmode
x          : Quantity on the x-axis. Must be a string.
y          : Quantity on the y-axis. Must be a string.
snap       : Number of the snapshot to plot. Defaults to `current'.
sim        : Number of the simulation to plot. Defaults to `current'.
overplot   : If True, overplots on the previous existing plot rather
\begin{quote}

than deleting it. Defaults to False.
\end{quote}
\begin{description}
\item[{autoscale}] \leavevmode{[}If True, the limits of the plot are set{]}
automatically.  Can also be set to `x' or `y' to specify
that only one of the axis has to use autoscaling.
If False (default), autoscaling is not used. On an axis that does
not have autoscaling turned on, global limits are used
if defined for the plotted quantity.

\item[{xunit}] \leavevmode{[}Specify the unit to use for the plotting for the quantity{]}
on the x-axis.

\item[{yunit}] \leavevmode{[}Specify the unit to use for the plotting for the quantity{]}
on the y-axis.

\item[{time}] \leavevmode{[}Plots the analytical solution for the given time.{]}
If not set, then reads the time from the sim or snapshot

\end{description}

\end{description}

\end{fulllineitems}

\index{time\_plot() (in module facade)}

\begin{fulllineitems}
\phantomsection\label{index:facade.time_plot}\pysiglinewithargsret{\code{facade.}\bfcode{time\_plot}}{\emph{x}, \emph{y}, \emph{sim='current'}, \emph{overplot=False}, \emph{autoscale=False}, \emph{xunit='default'}, \emph{yunit='default'}, \emph{xaxis='linear'}, \emph{yaxis='linear'}, \emph{idx=None}, \emph{idy=None}, \emph{id=None}, \emph{typex='default'}, \emph{typey='default'}, \emph{type='default'}, \emph{**kwargs}}{}
Plot two quantities as evolved in time one versus the another.  Creates
a new plotting window if one does not already exist.
\begin{description}
\item[{Required arguments:}] \leavevmode\begin{description}
\item[{x}] \leavevmode{[}Quantity on x-axis. Must be a string. The quantity is looked{]}
up in the quantities defined as a function of time. If it is
not found there, then we try to interpret it as a quantity
defined for a particle. In this case, the user needs to pass
either idx either id to specify which particle he wishes
to look-up.

\item[{y}] \leavevmode{[}Quantity on y-axis.  Must be a string. The interpretation is{]}
like for the previous argument.

\end{description}

\item[{Optional arguments:}] \leavevmode
sim        : Number of the simulation to plot. Defaults to `current'.
overplot   : If True, overplots on the previous existing plot rather
\begin{quote}

than deleting it. Defaults to False.
\end{quote}
\begin{description}
\item[{autoscale}] \leavevmode{[}If True, the limits of the plot are set{]}
automatically.  Can also be set to `x' or `y' to specify
that only one of the axis has to use autoscaling.
If False (default), autoscaling is not used. On an axis that
does not have autoscaling turned on, global limits are used
if defined for the plotted quantity.

\item[{xunit}] \leavevmode{[}Specify the unit to use for the plotting for the quantity{]}
on the x-axis.

\item[{yunit}] \leavevmode{[}Specify the unit to use for the plotting for the quantity{]}
on the y-axis.

\item[{idx}] \leavevmode{[}id of the particle to plot on the x-axis. Ignored if the{]}
quantity given (e.g., com\_x) does not depend on the id.

\end{description}

idy        : same as previous, on the y-axis.
id         : same as the two previous ones. To be used when the id is the
\begin{quote}

same on both axes. If set, overwrites the passed idx and idy.
\end{quote}
\begin{description}
\item[{typex}] \leavevmode{[}type of particles on the x-axis. Ignored if the quantity{]}
given does not depend on it

\end{description}

typey      : as the previous one, on the y-axis.
type       : as the previous ones, for both axis at the same time. If set,
\begin{quote}

overwrites typex and typey.
\end{quote}

\end{description}

\end{fulllineitems}

\index{savefig() (in module facade)}

\begin{fulllineitems}
\phantomsection\label{index:facade.savefig}\pysiglinewithargsret{\code{facade.}\bfcode{savefig}}{\emph{name}}{}
Saves the current figure with the given name.  Note that matplotlib
figures out automatically the type of the file from the extension.
\begin{description}
\item[{Required arguments:}] \leavevmode
name       : filename (including extension)

\end{description}

\end{fulllineitems}

\index{next() (in module facade)}

\begin{fulllineitems}
\phantomsection\label{index:facade.next}\pysiglinewithargsret{\code{facade.}\bfcode{next}}{}{}
Advances the current snapshot of the current simulation.
Return the new snapshot, or None if the call failed.

\end{fulllineitems}

\index{previous() (in module facade)}

\begin{fulllineitems}
\phantomsection\label{index:facade.previous}\pysiglinewithargsret{\code{facade.}\bfcode{previous}}{}{}
Decrements the current snapshot of the current simulation.
Return the new snapshot, or None if the call failed.

\end{fulllineitems}

\index{snap() (in module facade)}

\begin{fulllineitems}
\phantomsection\label{index:facade.snap}\pysiglinewithargsret{\code{facade.}\bfcode{snap}}{\emph{no}}{}
Jump to the given snapshot number of the current simulation.  Note that
you can use standard Numpy index notation (e.g., -1 is the last snapshot).
Return the new snapshot, or None if the call failed.
\begin{description}
\item[{Required arguments:}] \leavevmode
snapno     : Snapshot number

\end{description}

\end{fulllineitems}

\index{block() (in module facade)}

\begin{fulllineitems}
\phantomsection\label{index:facade.block}\pysiglinewithargsret{\code{facade.}\bfcode{block}}{}{}
Stops the execution flow until the user presses `enter'.
Useful in scripts, allowing to see a plot (which gets closed
when the execution flow reaches the end of the script

\end{fulllineitems}

\index{sims() (in module facade)}

\begin{fulllineitems}
\phantomsection\label{index:facade.sims}\pysiglinewithargsret{\code{facade.}\bfcode{sims}}{}{}
Print a list of the simulations to screen

\end{fulllineitems}

\index{snaps() (in module facade)}

\begin{fulllineitems}
\phantomsection\label{index:facade.snaps}\pysiglinewithargsret{\code{facade.}\bfcode{snaps}}{\emph{simno}}{}
For the given simulation number, print a list of all the snapshots
\begin{description}
\item[{Required argument:}] \leavevmode
simno      : Simulation number from which to print the snapshot list.

\end{description}

\end{fulllineitems}

\index{set\_current\_sim() (in module facade)}

\begin{fulllineitems}
\phantomsection\label{index:facade.set_current_sim}\pysiglinewithargsret{\code{facade.}\bfcode{set\_current\_sim}}{\emph{simno}}{}
Set the current simulation to the given number.
Returns the newly set current simulation.
\begin{description}
\item[{Required argument:}] \leavevmode
simno      : Simulation number

\end{description}

\end{fulllineitems}

\index{limit() (in module facade)}

\begin{fulllineitems}
\phantomsection\label{index:facade.limit}\pysiglinewithargsret{\code{facade.}\bfcode{limit}}{\emph{quantity}, \emph{min=None}, \emph{max=None}, \emph{auto=False}, \emph{window='current'}, \emph{subfigure='current'}}{}
Set plot limits. Quantity is the quantity to limit.
\begin{description}
\item[{Required arguments:}] \leavevmode
quantity   : Set limits of this variable. Must be a string.

\item[{Optional arguments:}] \leavevmode
min        : Minimum value of variable range.
max        : Maximum value of variable range.
auto       : If auto is set to True, then the limits for that quantity are
\begin{quote}

set automatically. Otherwise, use the one given by max and min.
\end{quote}
\begin{description}
\item[{window}] \leavevmode{[}If window is set to `global' is available, then any changes{]}
will affect also future plots that do not have autoscaling
turned on.

\item[{subfigure}] \leavevmode{[}If subfigure is set to `all', the limits in all the figures or{]}
in all the subfigures of the current figure are set.

\end{description}

\end{description}

\end{fulllineitems}

\index{make\_movie() (in module facade)}

\begin{fulllineitems}
\phantomsection\label{index:facade.make_movie}\pysiglinewithargsret{\code{facade.}\bfcode{make\_movie}}{\emph{filename}, \emph{snapshots='all'}, \emph{window\_no=0}, \emph{fps=24}}{}
Generates movie for plots generated in given window

\end{fulllineitems}

\index{KnownQuantities() (in module facade)}

\begin{fulllineitems}
\phantomsection\label{index:facade.KnownQuantities}\pysiglinewithargsret{\code{facade.}\bfcode{KnownQuantities}}{}{}
Return the list of the quantities that we know

\end{fulllineitems}

\index{get\_data() (in module facade)}

\begin{fulllineitems}
\phantomsection\label{index:facade.get_data}\pysiglinewithargsret{\code{facade.}\bfcode{get\_data}}{\emph{quantity}, \emph{snap='current'}, \emph{type='default'}, \emph{sim='current'}, \emph{unit='default'}}{}
Returns the array with the data for the given quantity.
The data is returned scaled to the specified unit
\begin{description}
\item[{Required argument:}] \leavevmode
quantity        :The quantity required. Must be a string

\item[{Optional arguments:}] \leavevmode
type            :The type of the particles (e.g. `star')
snap            :Number of the snapshot. Defaults to `current'
sim             :Number of the simulation. Defaults to `current'
unit            :Specifies the unit to use to return the data

\end{description}

\end{fulllineitems}

\index{get\_render\_data() (in module facade)}

\begin{fulllineitems}
\phantomsection\label{index:facade.get_render_data}\pysiglinewithargsret{\code{facade.}\bfcode{get\_render\_data}}{\emph{x}, \emph{y}, \emph{quantity}, \emph{sim='current'}, \emph{snap='current'}, \emph{renderunit='default'}, \emph{res=64}, \emph{zslice=None}, \emph{coordlimits=None}}{}
Return the rendered data for the given quantity. Useful when one needs
to grid SPH data. The result is scaled to the specified unit. The options are 
a subset of the options available to the `render' function.
\begin{description}
\item[{Required arguments:}] \leavevmode
x        : Quantity on the x-axis. Must be a string
y        : Quantity on the y-axis. Must be a string
quantity : Quantity to render.

\item[{Optional arguments:}] \leavevmode
snap     : Number of the snapshot to plot. Defaults to `current'.
sim      : Number of the simulation to plot. Defaults to `current'
renderunit: Unit to use for the rendered quantity
res      : Resolution
zslice   : z-coordinate of the slice when doing a slice rendering.
\begin{quote}

Default is None, which produces a column-integrated plot.
If you set this variable, a slice rendering will be
done instead.
\end{quote}
\begin{description}
\item[{coordlimits: Limits of the coordinates on x and y. See documentation}] \leavevmode
of render.

\end{description}

\item[{Return:}] \leavevmode
data     : The rendered data, scaled to the requested unit.

\end{description}

\end{fulllineitems}

\index{CreateTimeData() (in module facade)}

\begin{fulllineitems}
\phantomsection\label{index:facade.CreateTimeData}\pysiglinewithargsret{\code{facade.}\bfcode{CreateTimeData}}{\emph{name}, \emph{function}, \emph{*args}, \emph{**kwargs}}{}
Given a function that takes a snapshot as input, construct a 
FunctionTimeDataFetcher object from it and register it.
\begin{quote}

Return the FunctionTimeDataFetcher newly constructed.
\end{quote}

\end{fulllineitems}

\index{CreateUserQuantity() (in module facade)}

\begin{fulllineitems}
\phantomsection\label{index:facade.CreateUserQuantity}\pysiglinewithargsret{\code{facade.}\bfcode{CreateUserQuantity}}{\emph{name}, \emph{formula}, \emph{unitlabel='`}, \emph{unitname='`}, \emph{scaling\_factor=1}, \emph{label='`}}{}
Given a mathematical formula, build a data fetcher from it.
The quantity is given a name, which can now be used in plots and in other 
formulae.  When you construct a quantity, you can rely on one of the units we 
provide, in which case you can just pass as the scaling\_factor parameter the 
name of the unit you want inside the SimUnits class. For example, if your unit 
has dimensions of acceleration, you can pass `a' as the scaling\_factor 
parameter. Doing this allows the unit system to work seamlessly when plotting 
(i.e., you can specify the units you want the plot in).  Alternatively, you can 
build your own unit passing a numerical value for the scaling\_factor, 
a unitname and a latex label.  In this case, however, no rescaling is possible, 
as the unit system does not know how to rescale your unit.

\end{fulllineitems}

\index{window() (in module facade)}

\begin{fulllineitems}
\phantomsection\label{index:facade.window}\pysiglinewithargsret{\code{facade.}\bfcode{window}}{\emph{no=None}}{}
Changes the current window to the number specified. If the window
doesn't exist, recreate it.
\begin{description}
\item[{Required arguments:}] \leavevmode
winno      : Window number

\end{description}

\end{fulllineitems}

\index{subfigure() (in module facade)}

\begin{fulllineitems}
\phantomsection\label{index:facade.subfigure}\pysiglinewithargsret{\code{facade.}\bfcode{subfigure}}{\emph{nx}, \emph{ny}, \emph{current}}{}
Creates a subplot in the current window.
\begin{description}
\item[{Required arguments:}] \leavevmode
nx         : x-grid size
ny         : y-grid size
current    : id of active sub-figure.  If sub-figure already exists,
\begin{quote}

then this sets the new active sub-figure.
\end{quote}

\end{description}

\end{fulllineitems}

\index{switch\_nongui() (in module facade)}

\begin{fulllineitems}
\phantomsection\label{index:facade.switch_nongui}\pysiglinewithargsret{\code{facade.}\bfcode{switch\_nongui}}{}{}
Switches matplotlib backend, disabling interactive plotting.
Useful in scripts where no interaction is required

\end{fulllineitems}

\index{rescale() (in module facade)}

\begin{fulllineitems}
\phantomsection\label{index:facade.rescale}\pysiglinewithargsret{\code{facade.}\bfcode{rescale}}{\emph{quantity}, \emph{unitname}, \emph{window='current'}, \emph{subfig='current'}}{}
Rescales the specified quantity in the specified window to the specified unit
\begin{description}
\item[{Required arguments:}] \leavevmode
quantity   : Quantity to be rescaled.  Must be a string.
unitname   : Required unit for quantity.

\item[{Optional qrguments:}] \leavevmode
window     : Window containing plot
subfig     : Sub-figure in window containing plot

\end{description}

\end{fulllineitems}



\renewcommand{\indexname}{Python Module Index}
\begin{theindex}
\def\bigletter#1{{\Large\sffamily#1}\nopagebreak\vspace{1mm}}
\bigletter{f}
\item {\texttt{facade}}, \pageref{index:module-facade}
\end{theindex}

\renewcommand{\indexname}{Index}
\printindex
\end{document}
